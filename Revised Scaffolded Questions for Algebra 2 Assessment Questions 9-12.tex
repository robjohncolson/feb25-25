\documentclass[12pt]{article}

% Setting up page geometry
\usepackage[margin=1in]{geometry}

% Including packages for mathematical typesetting
\usepackage{amsmath}
\usepackage{amssymb}
\usepackage{mathtools}

% Including package for enhanced enumeration
\usepackage{enumitem}

% Including package for better spacing and formatting
\usepackage{parskip}

% Setting up font: Latin Modern
\usepackage{lmodern}

% Document begins
\begin{document}

% Creating title
\begin{center}
    \textbf{Revised Scaffolded Questions for Algebra 2 Assessment (Questions 9--12)}
\end{center}

% Introduction
This document provides revised scaffolded questions to help students prepare for questions 9 through 12 of the enVision Algebra 2 Progress Monitoring Assessment Form C. Each question includes four scaffolded steps to build understanding from basic concepts to the level required by the assessment, with clear guidance for concept-naive students.

% Section for Question 9
\section*{Question 9: Polynomial Long Division}
The original question asks to divide \( x^3 - 4x^2 + 6x - 2 \) by \( x - 1 \) and complete the quotient. The following questions build understanding of polynomial division.

\begin{enumerate}[label=9.\arabic*]
    \item \textbf{Basic Polynomial Division}: Divide each term by the divisor, matching powers of \( x \):
    \begin{enumerate}
        \item[a)] \( \frac{8x^3}{2x} = \_\_\_\_ \)
        \item[b)] \( \frac{10x^4 + 4x^2}{2x^2} = \frac{10x^4}{2x^2} + \frac{4x^2}{2x^2} = \_\_\_\_ + \_\_\_\_ \)
        \item[c)] Why divide term by term? \_\_\_\_\_\_\_\_\_\_\_\_
    \end{enumerate}
    \item \textbf{Simple Long Division}: Divide \( x^2 + 4x + 3 \) by \( x + 1 \):
    \begin{enumerate}
        \item[a)] \( x^2 \div x = \_\_\_\_ \), multiply: \( x(x + 1) = \_\_\_\_ \), subtract: \( (x^2 + 4x + 3) - (x^2 + x) = \_\_\_\_ \)
        \item[b)] Continue: \( 3x \div x = \_\_\_\_ \), multiply, subtract to get remainder 0.
        \item[c)] Result: \( x^2 + 4x + 3 = (x + 1)(\_\_\_\_) + \_\_\_\_ \)
    \end{enumerate}
    \item \textbf{Synthetic Division}: Use synthetic division for \( x^2 + 5x + 6 \) by \( x - 2 \):
    \begin{enumerate}
        \item[a)] Divisor \( x - 2 \), so use 2. Coefficients: 1, 5, 6. Setup: \\
        \[
        \begin{array}{r|rrrr}
        2 & 1 & 5 & 6 \\
          &   & \_\_ & \_\_ \\
        \hline
          & \_\_ & \_\_ & \_\_ \\
        \end{array}
        \]
        \item[b)] Quotient: \_\_\_\_, Remainder: \_\_\_\_
        \item[c)] Why is synthetic division faster for linear divisors? \_\_\_\_\_\_\_\_\_\_\_\_
    \end{enumerate}
    \item \textbf{Applying to the Original Problem}: Divide \( x^3 - 4x^2 + 6x - 2 \) by \( x - 1 \) using synthetic division:
    \begin{enumerate}
        \item[a)] Coefficients: \_\_\_\_, \_\_\_\_, \_\_\_\_, \_\_\_\_. Divisor: \( x - 1 \), so use \_\_\_\_.
        \item[b)] Perform synthetic division: \\
        \[
        \begin{array}{r|rrrr}
        1 & 1 & -4 & 6 & -2 \\
          &   & \_\_ & \_\_ & \_\_ \\
        \hline
          & \_\_ & \_\_ & \_\_ & \_\_ \\
        \end{array}
        \]
        \item[c)] Quotient: \_\_\_\_, Remainder: \_\_\_\_. Write as: \( x^3 - 4x^2 + 6x - 2 = (x - 1)(\_\_\_\_) + \_\_\_\_ \).
    \end{enumerate}
\end{enumerate}

% Section for Question 10
\section*{Question 10: Solving Literal Equations}
The original question asks to solve \( N = S(P - V) - F \) for the variable cost per unit \( V \). The following questions build understanding of solving literal equations.

\begin{enumerate}[label=10.\arabic*]
    \item \textbf{Simple Literal Equations}: Solve for the indicated variable, isolating it like solving for \( x \):
    \begin{enumerate}
        \item[a)] \( A = lw \), for \( l \): \( l = \_\_\_\_ \)
        \item[b)] \( P = 2l + 2w \), for \( w \): \( w = \_\_\_\_ \)
        \item[c)] Why isolate variables? \_\_\_\_\_\_\_\_\_\_\_\_
    \end{enumerate}
    \item \textbf{Equations with Grouping}: Solve:
    \begin{enumerate}
        \item[a)] \( y = m(x + b) \), for \( m \): \( m = \frac{y}{x + b} \)
        \item[b)] \( C = \pi d + k \), for \( d \): \( d = \_\_\_\_ \)
    \end{enumerate}
    \item \textbf{Business Context}: Solve profit-related formulas:
    \begin{enumerate}
        \item[a)] \( P = R - C \), for \( C \): \( C = \_\_\_\_ \)
        \item[b)] \( P = S(R - C) \), for \( R \): \( P = S R - S C \), so \( R = \_\_\_\_ \)
    \end{enumerate}
    \item \textbf{Applying to the Original Problem}: Given \( N = S(P - V) - F \), solve for \( V \):
    \begin{enumerate}
        \item[a)] Isolate the term with \( V \): \( N + F = S(P - V) \)
        \item[b)] Divide: \( \frac{N + F}{S} = P - V \)
        \item[c)] Solve: \( V = \_\_\_\_ \)
    \end{enumerate}
\end{enumerate}

% Section for Question 11
\section*{Question 11: Inverse Functions}
The original question asks for the inverse of \( f(x) = \sqrt{x - 10} \), representing years as a function of profits. The following questions build understanding of inverse functions.

\begin{enumerate}[label=11.\arabic*]
    \item \textbf{Inverse Function Basics}: If \( f(a) = b \), then \( f^{-1}(b) = a \). The inverse swaps \( x \) and \( y \)-coordinates:
    \begin{enumerate}
        \item[a)] If \( f(4) = 9 \), then \( f^{-1}(9) = \_\_\_\_ \)
        \item[b)] If \( f^{-1}(2) = 5 \), then \( f(5) = \_\_\_\_ \)
        \item[c)] Why swap \( x \) and \( y \)? \_\_\_\_\_\_\_\_\_\_\_\_
    \end{enumerate}
    \item \textbf{Linear Inverses}: Find the inverse of \( f(x) = 2x + 3 \):
    \begin{enumerate}
        \item[a)] Set \( y = 2x + 3 \), switch: \( x = 2y + 3 \)
        \item[b)] Solve: \( x - 3 = 2y \), so \( y = \_\_\_\_ \)
        \item[c)] Inverse: \( f^{-1}(x) = \_\_\_\_ \)
    \end{enumerate}
    \item \textbf{Square Root Inverses}: Find the inverse of \( f(x) = \sqrt{x - 4} \), \( x \geq 4 \):
    \begin{enumerate}
        \item[a)] Set \( y = \sqrt{x - 4} \), switch: \( x = \sqrt{y - 4} \)
        \item[b)] Solve: Square both sides: \( x^2 = y - 4 \), so \( y = \_\_\_\_ \)
        \item[c)] Inverse: \( f^{-1}(x) = x^2 + 4 \), for \( x \geq 0 \) (since \( y \geq 0 \)). Why the restriction? \_\_\_\_\_\_\_\_\_\_\_\_
    \end{enumerate}
    \item \textbf{Applying to the Original Problem}: For \( f(x) = \sqrt{x - 10} \), representing profit after \( x \) years:
    \begin{enumerate}
        \item[a)] Find inverse: Set \( y = \sqrt{x - 10} \), switch: \( x = \sqrt{y - 10} \), solve: \( y = \_\_\_\_ \)
        \item[b)] Inverse: \( f^{-1}(x) = \_\_\_\_ \), for \( x \geq 0 \). What does \( f^{-1}(x) \) represent? \_\_\_\_
        \item[c)] Compare to choices: \( (x - 10)^2 \), \( x^2 + 10 \), with domains \( x \geq 0 \) or \( x \geq -10 \).
    \end{enumerate}
\end{enumerate}

% Section for Question 12
\section*{Question 12: Average Rate of Change}
The original question asks for the average rate of change of \( f(x) = -2x^2 + 5 \) over \( -3.5 \leq x \leq 0 \). The following questions build understanding of average rate of change.

\begin{enumerate}[label=12.\arabic*]
    \item \textbf{Basic Average Rate of Change}: The average rate of change is the slope of the secant line: \( \frac{f(b) - f(a)}{b - a} \). For \( f(x) = 3x + 1 \), find from \( x = 1 \) to \( x = 3 \):
    \begin{enumerate}
        \item[a)] \( f(1) = \_\_\_\_ \), \( f(3) = \_\_\_\_ \)
        \item[b)] Rate: \( \frac{f(3) - f(1)}{3 - 1} = \_\_\_\_ \)
    \end{enumerate}
    \item \textbf{Quadratic Functions}: For \( f(x) = -x^2 + 2 \), find from \( x = -1 \) to \( x = 1 \):
    \begin{enumerate}
        \item[a)] \( f(-1) = \_\_\_\_ \), \( f(1) = \_\_\_\_ \)
        \item[b)] Rate: \( \frac{f(1) - f(-1)}{1 - (-1)} = \_\_\_\_ \)
    \end{enumerate}
    \item \textbf{Negative and Decimal Intervals}: For \( f(x) = -x^2 + 4 \), find from \( x = -2.5 \) to \( x = 0 \):
    \begin{enumerate}
        \item[a)] \( f(-2.5) = -(-2.5)^2 + 4 = \_\_\_\_ \)
        \item[b)] \( f(0) = \_\_\_\_ \)
        \item[c)] Rate: \( \frac{f(0) - f(-2.5)}{0 - (-2.5)} = \_\_\_\_ \)
    \end{enumerate}
    \item \textbf{Applying to the Original Problem}: For \( f(x) = -2x^2 + 5 \), find from \( x = -3.5 \) to \( x = 0 \):
    \begin{enumerate}
        \item[a)] \( f(-3.5) = -2(-3.5)^2 + 5 = \_\_\_\_ \)
        \item[b)] \( f(0) = \_\_\_\_ \)
        \item[c)] Rate: \( \frac{f(0) - f(-3.5)}{0 - (-3.5)} = \_\_\_\_ \). Compare to choices: 19.5, 7, -7, -19.5.
    \end{enumerate}
\end{enumerate}

% Ending the document
\end{document}