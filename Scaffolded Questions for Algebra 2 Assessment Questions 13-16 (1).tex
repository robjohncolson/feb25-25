\documentclass[12pt]{article}

% Setting up page geometry
\usepackage[margin=1in]{geometry}

% Including packages for mathematical typesetting
\usepackage{amsmath}
\usepackage{amssymb}
\usepackage{mathtools}

% Including package for enhanced enumeration
\usepackage{enumitem}

% Including package for better spacing and formatting
\usepackage{parskip}

% Setting up font: Latin Modern
\usepackage{lmodern}

% Document begins
\begin{document}

% Creating title
\begin{center}
    \textbf{Scaffolded Questions for Algebra 2 Assessment (Questions 13--16)}
\end{center}

% Introduction
This document provides scaffolded questions to help students prepare for questions 13 through 16 of the enVision Algebra 2 Progress Monitoring Assessment Form C. Each question includes four scaffolded steps to build understanding from basic concepts to the level required by the assessment.

% Section for Question 13
\section*{Question 13: Population Density and Radius}
The original question involves finding the delivery radius for a pizza restaurant to reach 30,000 people in a town with a population density of 1200 people per square mile. The following questions build understanding of area and radius calculations.

\begin{enumerate}[label=13.\arabic*]
    \item \textbf{Area of a Circle}: The area of a circle is given by \( A = \pi r^2 \). If a circular park has a radius of 3 miles, calculate its area (use \( \pi \approx 3.14 \)).
    \item \textbf{Population from Density}: A town has a population density of 1000 people per square mile. If a circular region has an area of 4 square miles, how many people live in that region? Use the formula:
    \[
    \text{Population} = \text{Density} \times \text{Area}.
    \]
    \item \textbf{Solving for Radius}: A circular delivery area needs to serve 12,000 people, and the population density is 1500 people per square mile. Find the area needed, then solve for the radius using \( A = \pi r^2 \) (use \( \pi \approx 3.14 \)).
    \item \textbf{Applying to the Original Problem}: A pizza restaurant wants to deliver to 30,000 people in a town with a population density of 1200 people per square mile. Calculate the necessary area, then find the radius of the delivery area. Round to one decimal place and compare to the choices: 2.8 miles, 5.0 miles, 1.6 miles, 8.0 miles.
\end{enumerate}

% Section for Question 14
\section*{Question 14: Simplifying Radicals and Exponents}
The original question asks to simplify \( \sqrt{8} + \sqrt{32} - 2^{\frac{3}{2}} \). The following questions build skills in simplifying radicals and exponential expressions.

\begin{enumerate}[label=14.\arabic*]
    \item \textbf{Simplifying a Single Radical}: Simplify \( \sqrt{18} \) by factoring the number under the square root into its prime factors and taking out pairs of factors.
    \item \textbf{Combining Like Radicals}: Simplify the expression \( \sqrt{12} + \sqrt{48} \). First, simplify each square root, then combine like terms.
    \item \textbf{Understanding Exponents}: Evaluate \( 3^{\frac{3}{2}} \). Rewrite the expression using the property \( a^{\frac{m}{n}} = \sqrt[n]{a^m} \), and compute the value.
    \item \textbf{Applying to the Original Expression}: Simplify \( \sqrt{8} + \sqrt{32} - 2^{\frac{3}{2}} \). Simplify each term: rewrite \( \sqrt{8} \) and \( \sqrt{32} \) in terms of \( \sqrt{2} \), compute \( 2^{\frac{3}{2}} \), and combine the results. Compare to the choices: \( -2\sqrt{2} - \sqrt[3]{2} \), \( 8\sqrt{2} \), \( 4\sqrt{2} \), 0.
\end{enumerate}

% Section for Question 15
\section*{Question 15: Inverse Variation}
The original question involves inverse variation where \( M \) varies inversely with \( x \), with \( M = 2 \) when \( x = 10 \), and asks for \( M \) when \( x = 5 \). The following questions build understanding of inverse variation.

\begin{enumerate}[label=15.\arabic*]
    \item \textbf{Understanding Inverse Variation}: If \( y \) varies inversely with \( x \), the relationship is \( y = \frac{k}{x} \). If \( y = 6 \) when \( x = 4 \), find the constant of variation \( k \).
    \item \textbf{Finding a New Value}: Using the relationship \( y = \frac{k}{x} \), with \( k = 12 \), calculate \( y \) when \( x = 3 \).
    \item \textbf{Setting Up the Equation}: If \( M \) varies inversely with \( x \), and \( M = 5 \) when \( x = 8 \), write the inverse variation equation by finding \( k \). Then, find \( M \) when \( x = 4 \).
    \item \textbf{Applying to the Original Problem}: Given \( M \) varies inversely with \( x \), and \( M = 2 \) when \( x = 10 \), find the constant \( k \). Then, calculate \( M \) when \( x = 5 \).
\end{enumerate}

% Section for Question 16
\section*{Question 16: Solving Logarithmic Equations}
The original question asks to solve \( -2 \ln (3x) = 5 \). The following questions build skills in solving equations involving natural logarithms.

\begin{enumerate}[label=16.\arabic*]
    \item \textbf{Understanding Logarithms}: If \( \ln(y) = 2 \), find \( y \). Use the fact that \( \ln(y) = c \) means \( y = e^c \).
    \item \textbf{Solving a Simple Log Equation}: Solve the equation \( \ln(x) = 3 \). Write the equation in exponential form and compute \( x \).
    \item \textbf{Handling Coefficients}: Solve the equation \( 2 \ln(x) = 4 \). First, isolate the logarithm by dividing both sides, then convert to exponential form to find \( x \).
    \item \textbf{Applying to the Original Equation}: Solve \( -2 \ln (3x) = 5 \). Divide both sides to isolate the logarithm, convert to exponential form, and solve for \( x \). Compare to the choices: 0.082, 0.027, 4.061, 36.547.
\end{enumerate}

% Ending the document
\end{document}