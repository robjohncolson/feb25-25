\documentclass[12pt]{article}

% Setting up page geometry
\usepackage[margin=1in]{geometry}

% Including packages for mathematical typesetting
\usepackage{amsmath}
\usepackage{amssymb}
\usepackage{mathtools}

% Including package for enhanced enumeration
\usepackage{enumitem}

% Including package for better spacing and formatting
\usepackage{parskip}

% Setting up font: Latin Modern
\usepackage{lmodern}

% For check boxes
\usepackage{wasysym} % For \Square and \XBox

% For tables (Q25 choices)
\usepackage{array}

% Document begins
\begin{document}

% Creating title
\begin{center}
    \textbf{Algebra 2 Assessment Review: Quadratics \& Complex Numbers}
\end{center}

This document provides revised scaffolded questions to help students prepare for questions 6, 8, 17, 20, 21, and 25 (Quadratics \& Complex Numbers group) of the enVision Algebra 2 Progress Monitoring Assessment Form C. Each question includes scaffolded steps to build understanding from basic concepts to the level required by the assessment, with clear guidance for concept-naive students. This is followed by the original assessment questions.

\section*{Scaffolded Review Questions}

% Section for Question 6
\subsection*{Scaffolded Question for Assessment Item 6: Solving Quadratic Equations with Complex Numbers}
The original question asks to solve \( -x^2 + 5x = 7 \) over complex numbers. The following questions build understanding of the quadratic formula and complex solutions.

\begin{enumerate}[label=6.\arabic*]
    \item \textbf{Complex Numbers}: The imaginary unit \( i \) satisfies \( i^2 = -1 \). Simplify:
    \begin{enumerate}[label=\alph*)]
        \item \( \sqrt{-16} = \sqrt{16} \cdot \sqrt{-1} = \underline{\hspace{1cm}} \)
        \item \( \sqrt{-36} = \underline{\hspace{1cm}} \)
        \item Why is \( \sqrt{-1} = i \)? \underline{\hspace{6cm}}
    \end{enumerate}
    \item \textbf{Quadratic Formula}: For \( ax^2 + bx + c = 0 \), solutions are \( x = \frac{-b \pm \sqrt{b^2 - 4ac}}{2a} \). Solve \( x^2 - 2x - 3 = 0 \):
    \begin{enumerate}[label=\alph*)]
        \item Identify: \( a = \underline{\hspace{0.5cm}} \), \( b = \underline{\hspace{0.5cm}} \), \( c = \underline{\hspace{0.5cm}} \)
        \item Discriminant: \( b^2 - 4ac = \underline{\hspace{1cm}} \). Is it positive, negative, or zero? \underline{\hspace{1cm}}
        \item Solutions: \( x = \frac{\underline{\hspace{0.5cm}} \pm \sqrt{\underline{\hspace{0.5cm}}}}{2} = \underline{\hspace{1.5cm}} \)
    \end{enumerate}
    \item \textbf{Complex Solutions}: Solve \( x^2 + 2x + 5 = 0 \):
    \begin{enumerate}[label=\alph*)]
        \item \( a = \underline{\hspace{0.5cm}} \), \( b = \underline{\hspace{0.5cm}} \), \( c = \underline{\hspace{0.5cm}} \)
        \item Discriminant: \( b^2 - 4ac = \underline{\hspace{1cm}} \). Since it’s negative, expect complex roots.
        \item Apply formula: \( x = \frac{-2 \pm \sqrt{-16}}{2} = \frac{-2 \pm 4i}{2} = \underline{\hspace{1.5cm}} \)
        \item Why does a negative discriminant mean complex roots? \underline{\hspace{4cm}}
    \end{enumerate}
    \item \textbf{Applying to the Original Problem}: Solve \( -x^2 + 5x = 7 \):
    \begin{enumerate}[label=\alph*)]
        \item Rewrite in standard form: \underline{\hspace{2.5cm}} = 0
        \item Identify: \( a = \underline{\hspace{0.5cm}} \), \( b = \underline{\hspace{0.5cm}} \), \( c = \underline{\hspace{0.5cm}} \)
        \item Discriminant: \( b^2 - 4ac = \underline{\hspace{1cm}} \)
        \item Solutions: \( x = \frac{\underline{\hspace{0.5cm}} \pm \sqrt{\underline{\hspace{0.5cm}}}}{\underline{\hspace{0.5cm}}} = \underline{\hspace{2.5cm}} \). Compare to choices.
    \end{enumerate}
\end{enumerate}

% Section for Question 8
\subsection*{Scaffolded Question for Assessment Item 8: Multiplying Complex Numbers}
The original question asks to simplify \( (i - 5)(3 + 2i) \). The following questions build understanding of complex number multiplication.

\begin{enumerate}[label=8.\arabic*]
    \item \textbf{Complex Number Basics}: Since \( i^2 = -1 \), simplify:
    \begin{enumerate}[label=\alph*)]
        \item \( i^2 = \underline{\hspace{1cm}} \)
        \item \( (2i)^2 = 4i^2 = \underline{\hspace{1cm}} \)
        \item Combine: \( 3 + 2i - 5i = \underline{\hspace{1.5cm}} \)
    \end{enumerate}
    \item \textbf{Simple Multiplication}: Multiply \( (1 + i)(2 + i) \):
    \begin{enumerate}[label=\alph*)]
        \item Use FOIL: \( (1)(2) + (1)(i) + (i)(2) + (i)(i) = \underline{\hspace{3cm}} \)
        \item Simplify: \( 2 + i + 2i + i^2 = 2 + 3i - 1 = \underline{\hspace{1.5cm}} \)
    \end{enumerate}
    \item \textbf{Practice with Larger Numbers}: Multiply \( (2 - i)(3 + 2i) \):
    \begin{enumerate}[label=\alph*)]
        \item FOIL: \( (2)(3) + (2)(2i) + (-i)(3) + (-i)(2i) = \underline{\hspace{3cm}} \)
        \item Simplify: \( 6 + 4i - 3i - 2i^2 = \underline{\hspace{1.5cm}} \)
    \end{enumerate}
    \item \textbf{Applying to the Original Problem}: Simplify \( (i - 5)(3 + 2i) \):
    \begin{enumerate}[label=\alph*)]
        \item FOIL: \( (i)(3) + (i)(2i) + (-5)(3) + (-5)(2i) = \underline{\hspace{3cm}} \)
        \item Simplify: \( 3i + 2i^2 - 15 - 10i = \underline{\hspace{3cm}} \)
        \item Combine: \( \underline{\hspace{1.5cm}} \). Compare to choices.
    \end{enumerate}
\end{enumerate}

% Section for Question 17
\subsection*{Scaffolded Question for Assessment Item 17: Factoring Quadratics and Finding Zeros}
The original question asks to factor \( x^2 - 33x + 32 \) to find the zeros of \( f(x) = x^2 - 33x + 32 \). The following questions build understanding of factoring quadratics.

\begin{enumerate}[label=17.\arabic*]
    \item \textbf{Basic Factoring}: Factor by finding two numbers that multiply to the constant term and add to the middle coefficient:
    \begin{enumerate}[label=\alph*)]
        \item \( x^2 + 7x + 10 \): Numbers multiply to 10, add to 7: 2, 5. \\
        Factored: \( (x + 2)(x + 5) \)
        \item \( x^2 - 9x + 20 \): Numbers multiply to 20, add to -9: -4, -5. \\
        Factored: \( (x - 4)(x - 5) \)
        \item Why does factoring find zeros? \underline{\hspace{6cm}}
    \end{enumerate}
    \item \textbf{Finding Zeros}: Find zeros by setting factors to zero:
    \begin{enumerate}[label=\alph*)]
        \item \( f(x) = (x - 3)(x + 6) \): Zeros: \( x = 3 \), \( x = -6 \)
        \item \( f(x) = (x - 2)(x - 8) \): Zeros: \( x = \underline{\hspace{1cm}} \), \( x = \underline{\hspace{1cm}} \)
        \item Verify one zero: For \( x = 2 \), compute \( f(2) = (2 - 2)(2 - 8) = \underline{\hspace{1cm}} \).
    \end{enumerate}
    \item \textbf{Larger Coefficients}: Factor \( x^2 - 14x + 45 \):
    \begin{enumerate}[label=\alph*)]
        \item Factors of 45: \( 1 \times 45 \), \( 3 \times 15 \), \( 5 \times 9 \). \\
        Add to -14: -5, -9. \\
        Factored: \( (x - 5)(x - 9) \)
        \item Zeros: \( x = 5 \), \( x = 9 \)
        \item Practice: Factor \( x^2 - 16x + 60 \): Numbers: \underline{\hspace{0.5cm}}, \underline{\hspace{0.5cm}}. \\
        Factored: \underline{\hspace{2cm}}. Zeros: \underline{\hspace{0.5cm}}, \underline{\hspace{0.5cm}}.
    \end{enumerate}
    \item \textbf{Applying to the Original Problem}: Factor \( x^2 - 33x + 32 \):
    \begin{enumerate}[label=\alph*)]
        \item Factors of 32: \( 1 \times 32 \), \( 2 \times 16 \), \( 4 \times 8 \). \\
        Add to -33: \underline{\hspace{1cm}}, \underline{\hspace{1cm}}.
        \item Factored: \( (x \underline{\hspace{1cm}})(x \underline{\hspace{1cm}}) \)
        \item Zeros: \( x = \underline{\hspace{1cm}} \), \( x = \underline{\hspace{1cm}} \)
    \end{enumerate}
\end{enumerate}

% Section for Question 20
\subsection*{Scaffolded Question for Assessment Item 20: Completing the Square}
The original question asks for the constant to add to both sides of \( 3x^2 + 4x = 5 \) to complete the square. The following questions build understanding of completing the square.

\begin{enumerate}[label=20.\arabic*]
    \item \textbf{Perfect Square Trinomials}: Complete to form a perfect square:
    \begin{enumerate}[label=\alph*)]
        \item \( x^2 + 10x + \underline{\hspace{1cm}} = (x + 5)^2 \): Half of 10: 5, squared: 25.
        \item \( x^2 - 6x + \underline{\hspace{1cm}} = (x - \underline{\hspace{0.5cm}})^2 \): Half of -6: -3, squared: 9.
        \item \( x^2 + 12x + \underline{\hspace{1cm}} = (x + \underline{\hspace{0.5cm}})^2 \): \underline{\hspace{1cm}}, \underline{\hspace{0.5cm}}.
    \end{enumerate}
    \item \textbf{Completing the Square (\( a = 1 \))}: For \( x^2 + 8x = 3 \):
    \begin{enumerate}[label=\alph*)]
        \item Half of 8: 4, squared: 16.
        \item Add: \( x^2 + 8x + 16 = 3 + 16 \).
        \item Factor: \( (x + 4)^2 = 19 \).
        \item Practice: For \( x^2 + 10x = 6 \): Constant: \underline{\hspace{1cm}}. Result: \underline{\hspace{2.5cm}}.
    \end{enumerate}
    \item \textbf{Completing the Square (\( a \neq 1 \))}: For \( 2x^2 + 12x = 8 \):
    \begin{enumerate}[label=\alph*)]
        \item Factor: \( 2(x^2 + 6x) = 8 \).
        \item Complete inside: Half of 6: 3, squared: 9. \\
        \( 2(x^2 + 6x + 9) = 8 + 2 \cdot 9 = 26 \).
        \item Simplify: \( 2(x + 3)^2 = 26 \).
        \item Constant added to right: \( 2 \cdot 9 = 18 \).
        \item Practice: For \( 4x^2 + 8x = 12 \): Constant: \underline{\hspace{1cm}}. Result: \underline{\hspace{2.5cm}}.
    \end{enumerate}
    \item \textbf{Applying to the Original Problem}: For \( 3x^2 + 4x = 5 \):
    \begin{enumerate}[label=\alph*)]
        \item Factor: \( 3(x^2 + \frac{4}{3}x) = 5 \).
        \item Complete: Half of \( \frac{4}{3} \): \( \frac{2}{3} \), squared: \( \frac{4}{9} \). \\
        \( 3\left(x^2 + \frac{4}{3}x + \frac{4}{9}\right) = 5 + 3 \cdot \frac{4}{9} \).
        \item Simplify: \( 3\left(x + \frac{2}{3}\right)^2 = 5 + \frac{12}{9} = 5 + \frac{4}{3} = \frac{19}{3} \).
        \item Constant added: \( 3 \cdot \frac{4}{9} = \frac{4}{3} \). Matches choice (B).
    \end{enumerate}
\end{enumerate}

% Section for Question 21
\subsection*{Scaffolded Question for Assessment Item 21: Complex Solutions to Quadratic Equations}
The original question asks to select the solutions to \( x^2 = -64 \). The following questions build understanding of complex solutions.

\begin{enumerate}[label=21.\arabic*]
    \item \textbf{Square Roots of Negative Numbers}: Since \( \sqrt{-1} = i \), \( \sqrt{-a} = i\sqrt{a} \) for positive \( a \). Simplify:
    \begin{enumerate}[label=\alph*)]
        \item \( \sqrt{-25} = \sqrt{25} \cdot \sqrt{-1} = 5i \)
        \item \( \sqrt{-36} = \underline{\hspace{1cm}} \)
        \item \( \sqrt{-100} = \underline{\hspace{1cm}} \)
        \item Why is \( \sqrt{-1} = i \)? \underline{\hspace{6cm}}
    \end{enumerate}
    \item \textbf{Simple Complex Solutions}: Solve:
    \begin{enumerate}[label=\alph*)]
        \item \( x^2 = -16 \): \( x = \pm \sqrt{-16} = \pm 4i \)
        \item \( x^2 = -49 \): \( x = \pm \underline{\hspace{1cm}} \)
        \item Verify: For \( x = 7i \), compute \( (7i)^2 = \underline{\hspace{1cm}} \).
    \end{enumerate}
    \item \textbf{Quadratic Formula}: Solve \( x^2 + 9 = 0 \):
    \begin{enumerate}[label=\alph*)]
        \item Rewrite: \( x^2 = -9 \)
        \item Quadratic formula: \( a = 1 \), \( b = 0 \), \( c = 9 \). \\
        \( x = \frac{0 \pm \sqrt{0 - 4(1)(9)}}{2} = \frac{\pm \sqrt{-36}}{2} = \frac{\pm 6i}{2} = \pm 3i \)
        \item Practice: Solve \( x^2 + 25 = 0 \): \( x = \pm \underline{\hspace{1cm}} \).
    \end{enumerate}
    \item \textbf{Applying to the Original Problem}: Solve \( x^2 = -64 \):
    \begin{enumerate}[label=\alph*)]
        \item Direct: \( x = \pm \sqrt{-64} = \pm 8i \)
        \item Quadratic formula: \( x^2 + 64 = 0 \), \( a = 1 \), \( b = 0 \), \( c = 64 \). \\
        \( x = \frac{\pm \sqrt{-256}}{2} = \frac{\pm 16i}{2} = \pm 8i \)
        \item Verify: \( (8i)^2 = 64i^2 = -64 \), \((-8i)^2 = 64i^2 = -64 \).
        \item Select solutions from: 8, \(-8i\), \(-8\), \(32i\), \(8i\), \(-32i\): \underline{\hspace{1cm}}, \underline{\hspace{1cm}}.
    \end{enumerate}
\end{enumerate}

% Section for Question 25
\subsection*{Scaffolded Question for Assessment Item 25: Quadratic Formula and Simplifying Radicals}
The original question asks to solve \( x^2 + 10x + 6 = 0 \) using the quadratic formula. The following questions build understanding of the quadratic formula and radical simplification.

\begin{enumerate}[label=25.\arabic*]
    \item \textbf{Identifying Coefficients}: For \( ax^2 + bx + c = 0 \), identify \( a, b, c \) to use in \( x = \frac{-b \pm \sqrt{b^2 - 4ac}}{2a} \):
    \begin{enumerate}[label=\alph*)]
        \item \( x^2 + 3x + 2 = 0 \): \( a = 1 \), \( b = 3 \), \( c = 2 \)
        \item \( 2x^2 - 5x + 1 = 0 \): \( a = \underline{\hspace{0.5cm}} \), \( b = \underline{\hspace{0.5cm}} \), \( c = \underline{\hspace{0.5cm}} \)
        \item Why identify coefficients? \underline{\hspace{6cm}}
    \end{enumerate}
    \item \textbf{Calculating Discriminant}: The discriminant \( b^2 - 4ac \) determines the number of roots (positive: two real, zero: one, negative: complex):
    \begin{enumerate}[label=\alph*)]
        \item \( x^2 + 4x + 3 = 0 \): \( b^2 - 4ac = 4^2 - 4(1)(3) = 16 - 12 = 4 \)
        \item \( x^2 + 6x + 2 = 0 \): \( b^2 - 4ac = \underline{\hspace{1cm}} - \underline{\hspace{1cm}} = \underline{\hspace{1cm}} \)
        \item What does a positive discriminant mean? \underline{\hspace{6cm}}
    \end{enumerate}
    \item \textbf{Simplifying Radicals}: Simplify square roots for the quadratic formula:
    \begin{enumerate}[label=\alph*)]
        \item \( \sqrt{28} = \sqrt{4 \cdot 7} = 2\sqrt{7} \)
        \item \( \sqrt{76} = \sqrt{4 \cdot 19} = \underline{\hspace{1cm}} \)
        \item Practice: \( \sqrt{80} = \underline{\hspace{1cm}} \). Why simplify radicals? \underline{\hspace{5cm}}
    \end{enumerate}
    \item \textbf{Applying to the Original Problem}: Solve \( x^2 + 10x + 6 = 0 \):
    \begin{enumerate}[label=\alph*)]
        \item Coefficients: \( a = 1 \), \( b = 10 \), \( c = 6 \)
        \item Discriminant: \( b^2 - 4ac = 10^2 - 4(1)(6) = 100 - 24 = 76 \)
        \item Simplify: \( \sqrt{76} = \sqrt{4 \cdot 19} = 2\sqrt{19} \)
        \item Solve: \( x = \frac{-10 \pm \sqrt{76}}{2} = \frac{-10 \pm 2\sqrt{19}}{2} = -5 \pm \sqrt{19} \)
    \end{enumerate}
\end{enumerate}

\section*{Original Assessment Questions}

\subsection*{Question 6}
Solve \( -x^2 + 5x = 7 \) over the set of complex numbers.
\begin{enumerate}[label=\Alph*.]
    \item \( \frac{5 \pm i\sqrt{3}}{2} \) , \( \frac{5 - i\sqrt{3}}{2} \)
    \item \( \frac{5 \pm i\sqrt{53}}{2} \) , \( \frac{5 - i\sqrt{53}}{2} \)
    \item \( \frac{-5 \pm i\sqrt{53}}{2} \) , \( \frac{-5 - i\sqrt{53}}{2} \) % OCR had 5-i... for second, corrected to -5-i...
    \item \( \frac{-5 \pm i\sqrt{3}}{2} \) , \( \frac{-5 - i\sqrt{3}}{2} \) % OCR had missing i, corrected.
\end{enumerate}
(Note: The options in the image are presented as single fractions. For clarity in LaTeX and to match the scaffold process, I've kept them as two separate roots for each choice if distinct, or as \( \pm \) if combined. The original options seemed to list \( \frac{A+B}{C}, \frac{A-B}{C} \). Option B in the image is the one where the discriminant would be -3, leading to \(5 \pm i\sqrt{3}\) over -2, which is \(\frac{-5 \mp i\sqrt{3}}{2}\). The actual answer is \( \frac{5 \pm i\sqrt{-3}}{-2} = \frac{5 \pm i\sqrt{3}}{-2} = \frac{-5 \mp i\sqrt{3}}{2} \). The closest choice, if there was a typo in the question and it was \(x^2-5x+7=0\), would be \( \frac{5 \pm i\sqrt{3}}{2} \) (Option A from image).
For \( -x^2+5x-7=0 \), \(x = \frac{-5 \pm \sqrt{25-28}}{-2} = \frac{-5 \pm \sqrt{-3}}{-2} = \frac{-5 \pm i\sqrt{3}}{-2} = \frac{5 \mp i\sqrt{3}}{2}\). This matches Option A in the image.)

\subsection*{Question 8}
Which of the following is equivalent to the expression \( (i - 5)(3 + 2i) \)?
\begin{enumerate}[label=\Alph*.]
    \item \( -7i - 13 \)
    \item \( 13i - 17 \)
    \item \( -7i - 17 \)
    \item \( -13i - 17 \)
\end{enumerate}

\subsection*{Question 17}
Factor the expression \( x^2 - 33x + 32 \) to reveal the zeros of the function defined by \( f(x) = x^2 - 33x + 32 \).
\begin{itemize}
    \item The factored expression is (\(x + \) \framebox[1cm]{\phantom{x}}) (\(x + \) \framebox[1cm]{\phantom{x}})
    \item The zeros of the function are \framebox[1cm]{\phantom{x}} and \framebox[1cm]{\phantom{x}}
\end{itemize}

\subsection*{Question 20}
What constant do you add to each side of the equation to solve by completing the square?
\[ 3x^2 + 4x = 5 \]
\begin{enumerate}[label=\Alph*.]
    \item \( \frac{9}{16} \) \qquad C. \( \frac{3}{2} \)
    \item \( \frac{4}{3} \) \qquad D. \( 6 \)
\end{enumerate}

\subsection*{Question 21}
Select the solutions of the equation \( x^2 = -64 \).
\begin{enumerate}[label=\Alph*.]
    \item[\XBox] 8 \qquad \qquad D. [\XBox] \(32i\)
    \item[\XBox] \(-8i\) \qquad \quad E. [\XBox] \(8i\)
    \item[\XBox] -8 \qquad \qquad F. [\XBox] \(-32i\)
\end{enumerate}
(Note: Replace \XBox with \Square if you want empty boxes for students to fill)

\subsection*{Question 25}
Solve \( x^2 + 10x + 6 = 0 \). Use the choices provided to complete the solution.
\begin{center}
\begin{tabular}{|c|c|c|c|}
\hline
10 & -10 & 5 & -5 \\
\hline
\(\sqrt{19}\) & \(\sqrt{6}\) & \(\sqrt{10}\) & \\
\hline
\end{tabular}
\end{center}
\( x = \) \framebox[1.5cm]{\phantom{x}} \( \pm \) \framebox[1.5cm]{\phantom{x}}

% Ending the document
\end{document}