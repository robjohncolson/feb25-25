\documentclass[12pt]{article}

% Setting up page geometry
\usepackage[margin=1in]{geometry}

% Including packages for mathematical typesetting
\usepackage{amsmath}
\usepackage{amssymb}
\usepackage{mathtools}

% Including package for enhanced enumeration
\usepackage{enumitem}

% Including package for better spacing and formatting
\usepackage{parskip}

% Setting up font: Latin Modern
\usepackage{lmodern}

% For check boxes
\usepackage{wasysym} % For \Square and \XBox

% For tables
\usepackage{array}

% Document begins
\begin{document}

% Creating title
\begin{center}
    \textbf{Algebra 2 Assessment Review: Probability \& Statistics}
\end{center}

This document provides revised scaffolded questions to help students prepare for questions 28, 36, 37, and 38 (Probability \& Statistics group) of the enVision Algebra 2 Progress Monitoring Assessment Form C. Each question includes scaffolded steps to build understanding from basic concepts to the level required by the assessment, with clear guidance for concept-naive students. This is followed by the original assessment questions.

\section*{Scaffolded Review Questions}

% Section for Question 28
\subsection*{Scaffolded Question for Assessment Item 28: Statistics Terminology}
The original question asks whether 45 (average points for the first 3 games) is a variable, parameter, sample, or statistic, given a season average of 42. The following questions build understanding of statistical terms.

\begin{enumerate}[label=28.\arabic*]
    \item \textbf{Population vs. Sample}:
    A \textbf{population} is the entire group of individuals or items that we are interested in studying.
    A \textbf{sample} is a subset of the population that is selected for study.
    \begin{enumerate}[label=\alph*)]
        \item Identify population and sample:
        Population: All basketball games in a season.
        Sample: The first 5 games of the season.
        \item Your example:
        Population: All students in a particular high school.
        Sample: \underline{All 10th-grade students in that high school (or, 50 randomly selected students from the school, etc.)}.
    \end{enumerate}
    \item \textbf{Parameter vs. Statistic}:
    A \textbf{parameter} is a numerical measure that describes a characteristic of the entire \textit{population}.
    A \textbf{statistic} is a numerical measure that describes a characteristic of a \textit{sample}.
    (Hint: \textbf{P}arameter for \textbf{P}opulation; \textbf{S}tatistic for \textbf{S}ample)
    \begin{enumerate}[label=\alph*)]
        \item Average score of all games played by a team in a season: 50 points. (This describes all games - the population). This is a \underline{parameter}.
        Average score of a sample of 10 games played by the team: 52 points. (This describes a sample). This is a \underline{statistic}.
        \item Average height of all students in a university: 5’6”. This is a \underline{parameter}.
        Average height of a sample of 30 students from that university: 5’7”. This is a \underline{statistic}.
    \end{enumerate}
    \item \textbf{Identifying Terms in Context}: Classify the given numbers:
    \begin{enumerate}[label=\alph*)]
        \item A town's mayor wants to know the average income of all households in the town. The true average income of all 2,500 households is \$55,000. This \$55,000 is a \underline{parameter}.
        The mayor surveys 100 households and finds their average income is \$52,000. This \$52,000 is a \underline{statistic}.
        \item A basketball coach calculates the average points scored by the team in the first 3 games of the season as 55 points. This 55 points is a \underline{statistic} (since it's based on a sample of games, not all games).
    \end{enumerate}
    \item \textbf{Applying to the Original Problem}:
    A high school basketball team had a season average of 42 points per game (this describes the entire season's games - the population of games for that season).
    For the first 3 games of the season (this is a sample of games), they averaged 45 points per game.
    \begin{enumerate}[label=\alph*)]
        \item Population: \underline{All games played by the team in that season}.
        Sample: \underline{The first 3 games of that season}.
        \item The value 42 (season average) describes the \underline{population}, so it is a \underline{parameter}.
        The value 45 (average of first 3 games) describes the \underline{sample}, so it is a \underline{statistic}.
        \item Practice: A company wants to know the average age of its 3000 employees. The true average age is 38 years. A researcher selects 100 employees and finds their average age is 37.5 years.
        The 37.5 years is a \underline{statistic}.
    \end{enumerate}
\end{enumerate}

% Section for Question 36
\subsection*{Scaffolded Question for Assessment Item 36: Set Operations and Probability}
The original question asks whether the winning outcomes (odd number or 6) are the union, intersection, or complement of \( A = \{1, 2, 3, 5, 6\} \) and \( B = \{1, 3, 4, 5, 6\} \) when rolling a number cube. (The sample space for a number cube is \(S = \{1, 2, 3, 4, 5, 6\}\)).

\begin{enumerate}[label=36.\arabic*]
    \item \textbf{Set Operations Review}:
    Let \( U = \{1, 2, 3, 4, 5, 6\} \) be the universal set.
    Let \( A = \{1, 3, 5\} \) (set of odd numbers).
    Let \( B = \{2, 4, 6\} \) (set of even numbers).
    Let \( C = \{4, 5, 6\} \).
    \begin{enumerate}[label=\alph*)]
        \item Union (\(\cup\)): \( A \cup C = \{x \mid x \in A \text{ OR } x \in C \text{ (or both)}\} = \{1, 3, 5\} \cup \{4, 5, 6\} = \underline{\{1, 3, 4, 5, 6\}} \).
        \item Intersection (\(\cap\)): \( A \cap C = \{x \mid x \in A \text{ AND } x \in C\} = \{1, 3, 5\} \cap \{4, 5, 6\} = \underline{\{5\}} \).
        \item Complement (\(A^c\) or \(A'\)): \( A^c = \{x \in U \mid x \notin A\} = \{1, 2, 3, 4, 5, 6\} - \{1, 3, 5\} = \underline{\{2, 4, 6\}} = B \).
        \item Why use union in probability for "OR" events? \underline{Union includes all outcomes that satisfy one event, the other event, or both, which is the meaning of "OR".}
    \end{enumerate}
    \item \textbf{Probability Context - "OR"}:
    A standard six-sided die is rolled. You win if you roll a 1 OR a 2.
    Let event \(E_1\) be rolling a 1: \( E_1 = \{1\} \).
    Let event \(E_2\) be rolling a 2: \( E_2 = \{2\} \).
    \begin{enumerate}[label=\alph*)]
        \item The set of winning outcomes is \( E_1 \cup E_2 = \{1\} \cup \{2\} = \underline{\{1, 2\}} \).
        \item If you win on an "odd number OR a multiple of 3":
        Odd numbers: \(O = \{1, 3, 5\}\). Multiples of 3: \(M_3 = \{3, 6\}\).
        Winning outcomes = \( O \cup M_3 = \{1, 3, 5\} \cup \{3, 6\} = \underline{\{1, 3, 5, 6\}} \).
    \end{enumerate}
    \item \textbf{Analyzing Specific Sets for "OR" Events}:
    Given Set \(X = \{1, 3, 5\}\) and Set \(Y = \{1, 2, 5\}\).
    \begin{enumerate}[label=\alph*)]
        \item Union: \( X \cup Y = \{1, 3, 5\} \cup \{1, 2, 5\} = \underline{\{1, 2, 3, 5\}} \).
        \item Intersection: \( X \cap Y = \{1, 3, 5\} \cap \{1, 2, 5\} = \underline{\{1, 5\}} \).
        \item Practice: If a game is won by an outcome in set X OR an outcome in set Y, which operation represents the winning outcomes? \underline{Union (\(\cup\))}.
    \end{enumerate}
    \item \textbf{Applying to the Original Problem}:
    Winning outcomes for Milianna: "an odd number OR 6".
    Let \(S = \{1, 2, 3, 4, 5, 6\}\) be the sample space for rolling a number cube.
    Set of odd numbers: \(O = \{1, 3, 5\}\).
    Set containing 6: \(S_6 = \{6\}\).
    \begin{enumerate}[label=\alph*)]
        \item The set of Milianna's winning outcomes is \( W = O \cup S_6 = \{1, 3, 5\} \cup \{6\} = \underline{\{1, 3, 5, 6\}} \).
        \item The question provides two sets: \( G_1 = \{1, 2, 3, 5, 6\} \) and \( G_2 = \{1, 3, 4, 5, 6\} \).
        We need to determine if Milianna's winning set \(W = \{1, 3, 5, 6\}\) is the union, intersection, or complement of \(G_1\) and \(G_2\).
        \item Calculate \(G_1 \cup G_2\): \(\{1, 2, 3, 5, 6\} \cup \{1, 3, 4, 5, 6\} = \{1, 2, 3, 4, 5, 6\}\). Is this \(W\)? \underline{No}.
        \item Calculate \(G_1 \cap G_2\): \(\{1, 2, 3, 5, 6\} \cap \{1, 3, 4, 5, 6\} = \underline{\{1, 3, 5, 6\}}\). Is this \(W\)? \underline{Yes}.
        \item (Complement usually needs a universal set defined by the problem context. If \(U = \{1,2,3,4,5,6\}\), \(G_1^c = \{4\}\), \(G_2^c = \{2\}\). Neither is \(W\).)
        \item Answer: Milianna's winning outcomes \(W=\{1,3,5,6\}\) are the \underline{intersection} of \(G_1\) and \(G_2\).
    \end{enumerate}
\end{enumerate}

% Section for Questions 37-38
\subsection*{Scaffolded Question for Assessment Items 37-38: Conditional Probability and Data Analysis}
The original questions involve calculating P(heavy metal | 12th grade) and comparing P(10th grade | rock) vs. P(rock | 10th grade) using a two-way table. The following questions build understanding of conditional probability.

\begin{center}
\textbf{Music Preference Data}
\begin{tabular}{l|ccc|c}
    & Rock & Hip-Hop & Heavy Metal & Total \\
    \hline
    10th Grade & 16 & 12 & 4 & 32 \\
    11th Grade & 18 & 10 & 12 & 40 \\
    12th Grade & 16 & 8 & 6 & 30 \\
    \hline
    Total & 50 & 30 & 22 & 102 \\
\end{tabular}
\end{center}

\begin{enumerate}[label=37-38.\arabic*]
    \item \textbf{Reading Two-Way Tables}:
    \begin{enumerate}[label=\alph*)]
        \item How many 11th graders prefer hip-hop? Look at the intersection of "11th Grade" row and "Hip-Hop" column: \underline{10}.
        \item What is the total number of 10th graders surveyed? Look at the total for the "10th Grade" row: \underline{32}.
        \item What is the total number of students who prefer Rock? \underline{50}.
        \item What is the grand total of students surveyed? \underline{102}.
        \item Why are totals important in probability? \underline{They often serve as the denominator (sample space size) for calculating probabilities.}
    \end{enumerate}
    \item \textbf{Basic Probability from a Table}: \( P(\text{Event}) = \frac{\text{Number of favorable outcomes for Event}}{\text{Total number of outcomes}} \).
    \begin{enumerate}[label=\alph*)]
        \item Probability a randomly selected student is in 11th grade: \( P(\text{11th grade}) = \frac{\text{Total 11th graders}}{\text{Grand Total}} = \frac{40}{102} = \underline{\frac{20}{51}} \).
        \item Probability a randomly selected student prefers Heavy Metal: \( P(\text{Heavy Metal}) = \frac{\text{Total Heavy Metal}}{\text{Grand Total}} = \frac{22}{102} = \underline{\frac{11}{51}} \).
    \end{enumerate}
    \item \textbf{Conditional Probability \( P(A|B) \)}: The probability of event A occurring GIVEN that event B has already occurred.
    Formula: \( P(A|B) = \frac{P(A \text{ and } B)}{P(B)} = \frac{\text{Number of outcomes in both A and B}}{\text{Number of outcomes in B}} \).
    When using a table, the condition B restricts our "new" sample space to only those outcomes in B.
    \begin{enumerate}[label=\alph*)]
        \item P(prefers Hip-Hop | student is in 10th grade):
        Our sample space is now only the 10th graders (Total = 32).
        Out of these 10th graders, how many prefer Hip-Hop? 12.
        So, \( P(\text{Hip-Hop } | \text{ 10th grade}) = \frac{12}{32} = \underline{\frac{3}{8}} \).
        \item P(is in 10th grade | student prefers Hip-Hop):
        Our sample space is now only students who prefer Hip-Hop (Total = 30).
        Out of these Hip-Hop preferrers, how many are in 10th grade? 12.
        So, \( P(\text{10th grade } | \text{ Hip-Hop}) = \frac{12}{30} = \underline{\frac{2}{5}} \).
        \item Compare \( P(\text{Hip-Hop } | \text{ 10th grade}) = \frac{3}{8} = 0.375 \) and \( P(\text{10th grade } | \text{ Hip-Hop}) = \frac{2}{5} = 0.4 \).
        Is \(P(A|B)\) generally equal to \(P(B|A)\)? \underline{No}.
    \end{enumerate}
    \item \textbf{Applying to the Original Problems}:
    \begin{enumerate}[label=\alph*)]
        \item \textbf{Question 37}: What is the probability that a randomly selected 12th grade student at the school favors heavy metal? This is \( P(\text{Heavy Metal } | \text{ 12th grade}) \).
        Condition: Student is in 12th grade. Total 12th graders = \underline{30}. (This is our new denominator).
        Out of these 12th graders, how many favor Heavy Metal? \underline{6}.
        \( P(\text{Heavy Metal } | \text{ 12th grade}) = \frac{6}{30} = \frac{1}{5} = 0.20 \).
        As a percentage: \( 0.20 \times 100\% = \underline{20\%} \).
        \item \textbf{Question 38}: Compare \(P(\text{10th grade } | \text{ Rock})\) and \(P(\text{Rock } | \text{ 10th grade})\).
        Calculate \(P(\text{10th grade } | \text{ Rock})\):
        Condition: Student chose Rock. Total Rock preferrers = \underline{50}.
        Number of 10th graders who chose Rock = \underline{16}.
        \(P(\text{10th grade } | \text{ Rock}) = \frac{16}{50} = \frac{8}{25} = \underline{0.32}\).

        Calculate \(P(\text{Rock } | \text{ 10th grade})\):
        Condition: Student is in 10th grade. Total 10th graders = \underline{32}.
        Number of 10th graders who chose Rock = \underline{16}.
        \(P(\text{Rock } | \text{ 10th grade}) = \frac{16}{32} = \frac{1}{2} = \underline{0.5}\).

        Compare: \(0.32\) versus \(0.5\).
        Since \(0.32 < 0.5\), then \(P(\text{10th grade } | \text{ Rock})\) is \underline{less than} \(P(\text{Rock } | \text{ 10th grade})\).
    \end{enumerate}
\end{enumerate}

\section*{Original Assessment Questions}

\subsection*{Question 28}
A high school basketball team had a season average of 42 points per game. For the first 3 games of the season, they averaged 45 points per game. Which word best describes the number 45?
\begin{enumerate}[label=\Alph*.]
    \item variable
    \item sample
    \item parameter
    \item statistic
\end{enumerate}

\subsection*{Question 36}
Milianna rolls a number cube and will win a game with an outcome of an odd number or 6. Complete the statement.
The winning outcomes are the
\begin{itemize}
    \item [\XBox] union
    \item [\XBox] intersection
    \item [\XBox] complement
    \item [\XBox] event
\end{itemize}
of \( \{1, 2, 3, 5, 6\} \) and \( \{1, 3, 4, 5, 6\} \).
(Note: Replace \XBox with \Square if you want empty boxes for students to fill)

\subsection*{Use the data in Items 37 and 38.}
The data show the favorite music of a random sample of students.
\begin{center}
\begin{tabular}{l|ccc|c}
    & Rock & Hip-Hop & Heavy Metal & Total \\
    \hline
    10th Grade & 16 & 12 & 4 & 32 \\
    11th Grade & 18 & 10 & 12 & 40 \\
    12th Grade & 16 & 8 & 6 & 30 \\
    \hline
    Total & 50 & 30 & 22 & 102 \\
\end{tabular}
\end{center}

\subsection*{Question 37}
What is the probability that a randomly selected 12th grade student at the school favors heavy metal?
\[ \framebox[2cm]{\phantom{Number}} \% \]

\subsection*{Question 38}
Complete the following to make a true statement.
The probability of randomly selecting a 10th grade student given the student chose rock is
\begin{itemize}
    \item [\XBox] greater than
    \item [\XBox] less than
    \item [\XBox] equal to
\end{itemize}
selecting a student who chose rock given the student is in 10th grade.
(Note: Replace \XBox with \Square if you want empty boxes for students to fill)

% Ending the document
\end{document}