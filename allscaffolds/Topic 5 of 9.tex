\documentclass[12pt]{article}

% Setting up page geometry
\usepackage[margin=1in]{geometry}

% Including packages for mathematical typesetting
\usepackage{amsmath}
\usepackage{amssymb}
\usepackage{mathtools}

% Including package for enhanced enumeration
\usepackage{enumitem}

% Including package for better spacing and formatting
\usepackage{parskip}

% Setting up font: Latin Modern
\usepackage{lmodern}

% For check boxes
\usepackage{wasysym} % For \Square and \XBox

% Document begins
\begin{document}

% Creating title
\begin{center}
    \textbf{Algebra 2 Assessment Review: Rational Functions \& Equations}
\end{center}

This document provides revised scaffolded questions to help students prepare for questions 34 and 35 (Rational Functions/Equations group) of the enVision Algebra 2 Progress Monitoring Assessment Form C. Each question includes scaffolded steps to build understanding from basic concepts to the level required by the assessment, with clear guidance for concept-naive students. This is followed by the original assessment questions.

\section*{Scaffolded Review Questions}

% Section for Question 34
\subsection*{Scaffolded Question for Assessment Item 34: Rational Equations and Extraneous Solutions}
The original question asks to solve \( \frac{x^2 + 4}{x - 1} = \frac{5}{x - 1} \) and identify extraneous solutions. The following questions build understanding of rational equations.

\begin{enumerate}[label=34.\arabic*]
    \item \textbf{Basic Rational Equations}: If denominators are equal, equate numerators:
    \begin{enumerate}[label=\alph*)]
        \item \( \frac{x + 2}{x - 3} = \frac{5}{x - 3} \): \( x + 2 = 5 \), \( x = 3 \). \\
        Check: \( x = 3 \) makes denominator zero (extraneous). No solution.
        \item \( \frac{2x}{x + 1} = \frac{4}{x + 1} \): \( 2x = 4 \), so \( x = \underline{\hspace{1cm}} \).
        Check: Does \(x=2\) make the denominator zero? \underline{\hspace{0.5cm}}. So, is \(x=2\) a valid solution? \underline{\hspace{0.5cm}}.
    \end{enumerate}
    \item \textbf{Extraneous Solutions}: Solutions making denominators zero are extraneous:
    \begin{enumerate}[label=\alph*)]
        \item For the equation \( \frac{x}{x - 4} = \frac{2}{x - 4} \): What value of \(x\) would make the denominator zero (and thus be an extraneous solution if it arises)? Extraneous if \( x = \underline{\hspace{1cm}} \).
        \item Solve \( \frac{x}{x - 4} = \frac{2}{x - 4} \): \( x = 2 \).
        Check: Is \(x=2\) the value that makes the denominator zero? \underline{\hspace{0.5cm}}. So, is \(x=2\) a valid solution or extraneous? \underline{\hspace{1.5cm}}.
    \end{enumerate}
    \item \textbf{Solving Equations}: Solve \( \frac{x^2 + 1}{x - 2} = \frac{3}{x - 2} \):
    \begin{enumerate}[label=\alph*)]
        \item Restriction: Denominator cannot be zero, so \( x - 2 \neq 0 \), which means \( x \neq 2 \).
        \item Equate numerators: \( x^2 + 1 = 3 \). Solve for \(x\): \( x^2 = 2 \), so \( x = \pm \sqrt{2} \).
        \item Check against restriction: Is \( \sqrt{2} = 2 \)? \underline{\hspace{0.5cm}}. Is \( -\sqrt{2} = 2 \)? \underline{\hspace{0.5cm}}.
        Are the solutions valid? \underline{\hspace{0.5cm}}.
    \end{enumerate}
    \item \textbf{Applying to the Original Problem}: Solve \( \frac{x^2 + 4}{x - 1} = \frac{5}{x - 1} \):
    \begin{enumerate}[label=\alph*)]
        \item Restriction: \( x - 1 \neq 0 \), so \( x \neq \underline{\hspace{1cm}} \).
        \item Equate numerators: \( x^2 + 4 = 5 \). Solve for \(x\): \( x^2 = \underline{\hspace{1cm}} \), so \( x = \pm \underline{\hspace{1cm}} \).
        \item Check against restriction:
        One potential solution is \( x = 1 \). Is this the restricted value? \underline{\hspace{0.5cm}}. So, \( x = 1 \) is \underline{\hspace{1.5cm}}.
        The other potential solution is \( x = -1 \). Is this the restricted value? \underline{\hspace{0.5cm}}. So, \( x = -1 \) is \underline{\hspace{1.5cm}}.
        \item Answer: Valid solution is \( x = \underline{\hspace{1cm}} \). Extraneous solution is \( x = \underline{\hspace{1cm}} \).
    \end{enumerate}
\end{enumerate}

% Section for Question 35
\subsection*{Scaffolded Question for Assessment Item 35: Discontinuities in Rational Functions}
The original question asks where discontinuities occur in \( f(x) = \frac{x^2 + 5x}{x^2 - 2x - 35} \). The following questions build understanding of discontinuities.

\begin{enumerate}[label=35.\arabic*]
    \item \textbf{Factoring Quadratics}: Factor to find zeros (roots):
    \begin{enumerate}[label=\alph*)]
        \item \( x^2 - 6x + 8 = (x - 2)(x - 4) \): Zeros: \( x = 2, 4 \)
        \item \( x^2 + 3x - 10 \): Find two numbers that multiply to -10 and add to 3. Numbers: \underline{5}, \underline{-2}.
        Factored form: \((x + \underline{\hspace{0.5cm}})(x - \underline{\hspace{0.5cm}})\). Zeros: \underline{\hspace{1cm}}, \underline{\hspace{1cm}}.
    \end{enumerate}
    \item \textbf{Finding Discontinuities}: Discontinuities occur where the denominator of a rational function equals 0 (because division by zero is undefined):
    \begin{enumerate}[label=\alph*)]
        \item \( f(x) = \frac{x}{x - 5} \): Denominator is \(x-5\). Set \(x-5=0\). Discontinuity at \( x = 5 \).
        \item \( f(x) = \frac{1}{x^2 - 4} \): Denominator is \(x^2-4\). Factor it: \((x-2)(x+2)\).
        Set denominator to zero: \((x-2)(x+2)=0\). Discontinuities at \( x = \underline{\hspace{1cm}} \), \( x = \underline{\hspace{1cm}} \).
        \item Why do discontinuities occur where the denominator is zero? \underline{\hspace{6cm}}
    \end{enumerate}
    \item \textbf{Removable Discontinuities (Holes) vs. Vertical Asymptotes}: If a factor \((x-c)\) in the denominator cancels with a factor in the numerator, there is a removable discontinuity (a hole) at \(x=c\). If it doesn't cancel, it's usually a vertical asymptote. However, both are types of discontinuities.
    \begin{enumerate}[label=\alph*)]
        \item \( f(x) = \frac{x - 1}{x^2 - x} = \frac{x - 1}{x(x - 1)} \).
        Denominator zeros at \(x=0\) and \(x=1\).
        The factor \((x-1)\) cancels.
        Discontinuity at \(x=0\) (vertical asymptote).
        Discontinuity at \(x=1\) (removable discontinuity/hole).
        \item \( f(x) = \frac{x + 2}{x^2 + 5x + 6} \).
        Factor denominator: \(x^2 + 5x + 6 = (x+2)(x+3)\).
        So, \( f(x) = \frac{x+2}{(x+2)(x+3)} \).
        Denominator zeros at \(x = \underline{-2}\) and \(x = \underline{-3}\). These are the locations of discontinuities.
        At \(x=-2\), the factor \((x+2)\) cancels, so it's a \underline{hole (removable)}.
        At \(x=-3\), the factor \((x+3)\) does not cancel, so it's a \underline{vertical asymptote}.
        Discontinuities at: \underline{\hspace{1cm}}, \underline{\hspace{1cm}}.
    \end{enumerate}
    \item \textbf{Applying to the Original Problem}: For \( f(x) = \frac{x^2 + 5x}{x^2 - 2x - 35} \):
    \begin{enumerate}[label=\alph*)]
        \item Factor the numerator: \( x^2 + 5x = x(x + 5) \). Zeros of numerator: \( x = 0, -5 \).
        \item Factor the denominator: \( x^2 - 2x - 35 \). Find two numbers that multiply to -35 and add to -2. Numbers: \underline{-7}, \underline{5}.
        Denominator: \((x - 7)(x + 5) \). Zeros of denominator: \( x = \underline{\hspace{1cm}} \), \( x = \underline{\hspace{1cm}} \).
        \item Discontinuities occur where the original denominator is zero.
        So, discontinuities are at \( x = 7 \) and \( x = -5 \).
        At \( x = -5 \), the factor \( (x + 5) \) cancels from numerator and denominator. This means there is a \underline{hole (removable discontinuity)} at \( x = -5 \).
        At \( x = 7 \), the factor \( (x - 7) \) does not cancel. This means there is a \underline{vertical asymptote} at \( x = 7 \).
        The question asks "Where will the discontinuities occur". Both types are discontinuities.
        Answer: \( x = \underline{-5} \), \( x = \underline{7} \).
    \end{enumerate}
\end{enumerate}

\section*{Original Assessment Questions}

\subsection*{Question 34}
Use the equation \( \frac{x^2 + 4}{x - 1} = \frac{5}{x - 1} \) to answer the questions.

\textbf{Part A}
Solve the equation for \(x\).
\[ x = \framebox[1.5cm]{\phantom{X}} \]

\textbf{Part B}
Are there any extraneous solutions? Explain why or why not.
\begin{enumerate}[label=\Alph*.]
    \item There are no extraneous solutions because all solutions are real numbers.
    \item \( x = 1 \) is an extraneous solution because it makes a denominator equal to 0.
    \item \( x = -1 \) is an extraneous solution because it makes a denominator equal to 0.
    \item \( x = 0 \) is an extraneous solution because zero can not be a solution.
\end{enumerate}

\subsection*{Question 35}
Where will the discontinuities occur in the graph of the rational function?
\[ f(x) = \frac{x^2 + 5x}{x^2 - 2x - 35} \]
\begin{enumerate}[label=\Alph*.]
    \item at \( x = -5 \)
    \item at \( x = 7 \)
    \item at \( x = 0 \), \( x = -5 \) and \( x = 7 \)
    \item at \( x = -5 \) and \( x = 7 \)
\end{enumerate}

% Ending the document
\end{document}