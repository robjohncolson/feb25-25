\documentclass[12pt]{article}

% Setting up page geometry
\usepackage[margin=1in]{geometry}

% Including packages for mathematical typesetting
\usepackage{amsmath}
\usepackage{amssymb}
\usepackage{mathtools}

% Including package for enhanced enumeration
\usepackage{enumitem}

% Including package for better spacing and formatting
\usepackage{parskip}

% Setting up font: Latin Modern
\usepackage{lmodern}

% For check boxes
\usepackage{wasysym} % For \Square and \XBox

% Document begins
\begin{document}

% Creating title
\begin{center}
    \textbf{Algebra 2 Assessment Review: Sequences}
\end{center}

This document provides revised scaffolded questions to help students prepare for question 18 (Sequences group) of the enVision Algebra 2 Progress Monitoring Assessment Form C. Each question includes scaffolded steps to build understanding from basic concepts to the level required by the assessment, with clear guidance for concept-naive students. This is followed by the original assessment questions.

\section*{Scaffolded Review Questions}

% Section for Question 18
\subsection*{Scaffolded Question for Assessment Item 18: Arithmetic Sequences}
The original question involves determining if the sequence (Monday: 240, Tuesday: 290, Friday: 440) is arithmetic and predicting Saturday’s attendance. The following questions build understanding of arithmetic sequences.

\begin{enumerate}[label=18.\arabic*]
    \item \textbf{Identifying Arithmetic Sequences}: A sequence is arithmetic if the differences between consecutive terms are constant (this constant difference is called the common difference, \(d\)):
    \begin{enumerate}[label=\alph*)]
        \item \( 4, 7, 10, 13, \ldots \)
        Differences: \( 7 - 4 = 3 \), \( 10 - 7 = 3 \), \( 13 - 10 = 3 \).
        Arithmetic? \underline{Yes}. Common difference: \( d = 3 \).
        \item \( 8, 6, 4, 2, \ldots \)
        Differences: \( 6 - 8 = \underline{-2} \), \( 4 - 6 = \underline{-2} \), \( 2 - 4 = \underline{-2} \).
        Arithmetic? \underline{Yes}. Common difference: \( d = \underline{-2} \).
        \item Why must differences be constant for an arithmetic sequence? \underline{That is the definition of an arithmetic sequence; it has a constant rate of change.}
    \end{enumerate}
    \item \textbf{Finding Common Differences from Context}: Given festival attendance:
    \begin{enumerate}[label=\alph*)]
        \item Monday = 200, Tuesday = 250: \( d = 250 - 200 = \underline{50} \)
        \item Monday = 240, Tuesday = 290: \( d = 290 - 240 = \underline{50} \)
        \item If Wednesday's attendance based on the trend from (b) is 340, check the difference: \( 340 - 290 = \underline{50} \). Is it consistent with the previous difference? \underline{Yes}.
    \end{enumerate}
    \item \textbf{Recursive Formulas}: For an arithmetic sequence, the recursive formula is \( a_n = a_{n-1} + d \), where \(a_1\) is the first term and \(a_n\) is the \(n\)-th term.
    \begin{enumerate}[label=\alph*)]
        \item Sequence: \( 5, 9, 13, 17, \ldots \)
        First term: \( a_1 = 5 \). Common difference: \( d = 9-5 = 4 \).
        Recursive Formula: \( a_1 = 5 \), \( a_n = a_{n-1} + 4 \) (for \(n > 1\)).
        \item Sequence representing attendance: 240 (Monday), 290 (Tuesday), 340 (Wednesday, if arithmetic), ...
        First term (Monday's attendance): \( a_1 = \underline{240} \).
        Common difference (if arithmetic): \( d = \underline{50} \).
        Recursive Formula: \( a_1 = \underline{240} \), \( a_n = a_{n-1} + \underline{50} \) (for \(n > 1\)).
    \end{enumerate}
    \item \textbf{Applying to the Original Problem}:
    Monday attendance (\(a_1\)) = 240.
    Tuesday attendance (\(a_2\)) = 290.
    Friday attendance (\(a_5\)) = 440. (Monday is day 1, Tuesday day 2, Wed day 3, Thurs day 4, Friday day 5)
    \begin{enumerate}[label=\alph*)]
        \item \textbf{Part A: Is it arithmetic? Check recursive formula.}
        Calculate potential common difference from Monday to Tuesday: \( d = a_2 - a_1 = 290 - 240 = \underline{50} \).
        If the sequence is arithmetic with \(d=50\):
        Wednesday (\(a_3\)) = \(a_2 + d = 290 + 50 = 340 \).
        Thursday (\(a_4\)) = \(a_3 + d = 340 + 50 = 390 \).
        Friday (\(a_5\)) = \(a_4 + d = 390 + 50 = 440 \).
        Does this calculated Friday attendance match the given Friday attendance (440)? \underline{Yes}.
        So, is the sequence arithmetic? \underline{Yes}.
        Recursive formula: \( a_1 = \underline{240} \), \( a_n = a_{n-1} + \underline{50} \) (for \(n>1\)). (This matches option C from the assessment).
        \item \textbf{Part B: Predict Saturday's attendance.}
        Saturday is the 6th day (\(a_6\)).
        Using the recursive formula: \(a_6 = a_5 + d\).
        \(a_6 = 440 + 50 = \underline{490} \) people.
    \end{enumerate}
\end{enumerate}

\section*{Original Assessment Question}

\subsection*{Question 18}
The number of people attending a music festival has been increasing over the last several days. On Monday, 240 people attended. On Tuesday, 290 people attended. And on Friday, 440 people attended.

\textbf{Part A} Is the sequence that represents the festival attendance arithmetic? If it is, choose the recursive formula for the sequence.
\begin{enumerate}[label=\Alph*.]
    \item No; the music festival attendance cannot be represented by an arithmetic sequence.
    \item Yes; \( a(n) = 290 + n \)
    \item Yes; \( a_1 = 240, a_n = a_{n-1} + 50 \)
    \item Yes; \( a_1 = 240, a_n = a_{n+1} + 50 \)
\end{enumerate}

\textbf{Part B} If the trend continues, how many people will attend on Saturday?
\[ \framebox[2cm]{\phantom{Number}} \text{ people} \]

% Ending the document
\end{document}