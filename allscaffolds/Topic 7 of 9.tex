\documentclass[12pt]{article}

% Setting up page geometry
\usepackage[margin=1in]{geometry}

% Including packages for mathematical typesetting
\usepackage{amsmath}
\usepackage{amssymb}
\usepackage{mathtools}

% Including package for enhanced enumeration
\usepackage{enumitem}

% Including package for better spacing and formatting
\usepackage{parskip}

% Setting up font: Latin Modern
\usepackage{lmodern}

% For check boxes
\usepackage{wasysym} % For \Square and \XBox

% For pi symbol
\usepackage{textcomp} % for \textpi

% Document begins
\begin{document}

% Creating title
\begin{center}
    \textbf{Algebra 2 Assessment Review: Problem Solving, Rates, \& Miscellaneous Algebra}
\end{center}

This document provides revised scaffolded questions to help students prepare for questions 3, 10, 13, 14, 15, and 32 (Problem Solving/Rates/Misc. group) of the enVision Algebra 2 Progress Monitoring Assessment Form C. Each question includes scaffolded steps to build understanding from basic concepts to the level required by the assessment, with clear guidance for concept-naive students. This is followed by the original assessment questions.

\section*{Scaffolded Review Questions}

% Section for Question 3
\subsection*{Scaffolded Question for Assessment Item 3: Work Rate Problems}
The original question involves two faucets filling a tank together, one taking 8 hours and the other 4 hours. The following questions build understanding of work rates.

\begin{enumerate}[label=3.\arabic*]
    \item \textbf{Understanding Rates}: If a task (e.g., filling a tank) takes \( t \) hours to complete, the rate of work is \( \frac{1}{t} \) of the task per hour.
    \begin{enumerate}[label=\alph*)]
        \item Faucet takes 5 hours to fill 1 tank: Rate = \underline{\(\frac{1}{5}\)} tank/hour.
        \item Faucet takes 10 hours to fill 1 tank: Rate = \underline{\(\frac{1}{10}\)} tank/hour.
        \item Why is the rate the reciprocal of time? \underline{Rate measures how much of the job is done per unit of time. If the whole job (1) takes \(t\) time, then in 1 unit of time, \(1/t\) of the job is done.}
    \end{enumerate}
    \item \textbf{Combining Rates}: When two entities work together, their individual rates add up to the combined rate.
    Faucet A takes 6 hours (rate \(R_A = \frac{1}{6}\) tank/hour).
    Faucet B takes 12 hours (rate \(R_B = \frac{1}{12}\) tank/hour).
    \begin{enumerate}[label=\alph*)]
        \item Combined rate \(R_C = R_A + R_B = \frac{1}{6} + \frac{1}{12} = \frac{2}{12} + \frac{1}{12} = \underline{\frac{3}{12}} = \frac{1}{4}\) tank/hour.
        \item Time to fill together (\(t_C\)): If \(R_C = \frac{1}{t_C}\), then \(t_C = \frac{1}{R_C}\).
        \(t_C = \frac{1}{(1/4)} = \underline{4}\) hours.
    \end{enumerate}
    \item \textbf{Setting Up the Equation}: For Faucet A (takes \( a \) hours) and Faucet B (takes \( b \) hours), the combined time \( t \) satisfies: \( \frac{1}{a} + \frac{1}{b} = \frac{1}{t} \).
    \begin{enumerate}[label=\alph*)]
        \item Faucet A: 10 hours (\(a=10\)), Faucet B: 5 hours (\(b=5\)).
        Equation: \underline{\(\frac{1}{10} + \frac{1}{5} = \frac{1}{t}\)}.
        \item Solve the equation from part a:
        Combined rate = \( \frac{1}{10} + \frac{1}{5} = \frac{1}{10} + \frac{2}{10} = \underline{\frac{3}{10}} \) tank/hour.
        So, \( \frac{1}{t} = \frac{3}{10} \implies t = \underline{\frac{10}{3}}\) hours.
        \( \frac{10}{3} \text{ hours} = 3 \frac{1}{3} \text{ hours} = 3 \text{ hours and } \frac{1}{3} \times 60 = 20 \text{ minutes}\).
    \end{enumerate}
    \item \textbf{Applying to the Original Problem}: Faucet A takes 8 hours, Faucet B takes 4 hours.
    \begin{enumerate}[label=\alph*)]
        \item Rates: Faucet A: \(R_A = \underline{\frac{1}{8}}\) tank/hour, Faucet B: \(R_B = \underline{\frac{1}{4}}\) tank/hour.
        \item Combined rate: \( R_C = \frac{1}{8} + \frac{1}{4} = \frac{1}{8} + \frac{2}{8} = \underline{\frac{3}{8}} \) tank/hour.
        \item Time to fill together (\(t\)): \( t = \frac{1}{R_C} = \frac{1}{(3/8)} = \underline{\frac{8}{3}} \) hours.
        \item Convert to hours and minutes: \( \frac{8}{3} \text{ hours} = 2 \frac{2}{3} \text{ hours}\).
        Whole hours = \underline{2} hours.
        Fraction of an hour = \( \frac{2}{3} \) hours. Convert to minutes: \( \frac{2}{3} \times 60 = \underline{40} \) minutes.
        Total time: 2 hours and 40 minutes.
    \end{enumerate}
\end{enumerate}

% Section for Question 10
\subsection*{Scaffolded Question for Assessment Item 10: Solving Literal Equations}
The original question asks to solve \( N = S(P - V) - F \) for the variable cost per unit \( V \). The following questions build understanding of solving literal equations.

\begin{enumerate}[label=10.\arabic*]
    \item \textbf{Simple Literal Equations}: Solve for the indicated variable by isolating it, just like solving for \( x \) in a regular equation.
    \begin{enumerate}[label=\alph*)]
        \item \( A = lw \), for \( l \): \( l = \underline{\frac{A}{w}} \)
        \item \( P = 2l + 2w \), for \( w \):
        \( P - 2l = 2w \)
        \( w = \underline{\frac{P - 2l}{2}} \) or \( w = \underline{\frac{P}{2} - l} \)
        \item Why isolate variables? \underline{To express one quantity in terms of others, or to find its value if other values are known.}
    \end{enumerate}
    \item \textbf{Equations with Grouping (Parentheses or Fractions)}: Solve:
    \begin{enumerate}[label=\alph*)]
        \item \( y = m(x + b) \), for \( m \): \( m = \frac{y}{x + b} \) (assuming \(x+b \neq 0\))
        \item \( C = \pi d + k \), for \( d \):
        \( C - k = \pi d \)
        \( d = \underline{\frac{C-k}{\pi}} \) (assuming \(\pi \neq 0\))
    \end{enumerate}
    \item \textbf{Business Context Example}: Solve profit-related formulas:
    \begin{enumerate}[label=\alph*)]
        \item \( \text{Profit} = \text{Revenue} - \text{Cost} \) (\( P = R - C \)), for \( C \): \( C = \underline{R - P} \)
        \item \( \text{Net Income} = \text{Sales Volume} \times (\text{Revenue per unit} - \text{Cost per unit}) \) (\( N = S(R - C) \)), for \( R \):
        First, distribute \(S\): \( N = SR - SC \)
        Add \(SC\) to both sides: \( N + SC = SR \)
        Divide by \(S\): \( R = \underline{\frac{N + SC}{S}} \) or \( R = \underline{\frac{N}{S} + C} \) (assuming \(S \neq 0\))
        Alternatively, divide by S first: \(\frac{N}{S} = R-C\), then \(R = \frac{N}{S} + C\).
    \end{enumerate}
    \item \textbf{Applying to the Original Problem}: Given \( N = S(P - V) - F \), solve for \( V \):
    \begin{enumerate}[label=\alph*)]
        \item Add \(F\) to both sides to isolate the term with \( V \): \( N + F = S(P - V) \)
        \item Divide by \(S\) (assuming \(S \neq 0\)): \( \frac{N + F}{S} = P - V \)
        \item Add \(V\) to both sides: \( V + \frac{N + F}{S} = P \)
        Subtract \( \frac{N + F}{S} \) from both sides: \( V = \underline{P - \frac{N + F}{S}} \)
        Alternatively from step b: \(P-V = \frac{N+F}{S}\). Then \(-V = \frac{N+F}{S} - P\). So \(V = -\left(\frac{N+F}{S} - P\right) = P - \frac{N+F}{S}\).
    \end{enumerate}
\end{enumerate}

% Section for Question 13
\subsection*{Scaffolded Question for Assessment Item 13: Population Density and Radius}
The original question involves finding the delivery radius for a pizza restaurant to reach 30,000 people in a town with a population density of 1200 people per square mile. The following questions build understanding of area and radius calculations. (Use \( \pi \approx 3.14 \) or the \(\pi\) button on a calculator).

\begin{enumerate}[label=13.\arabic*]
    \item \textbf{Area of a Circle}: The area of a circle is given by \( A = \pi r^2 \), where \(r\) is the radius.
    If a circular park has a radius of 3 miles, calculate its area:
    \( A = \pi (3 \text{ miles})^2 = 9\pi \text{ miles}^2 \approx 9 \times 3.14159 = \underline{28.27} \text{ miles}^2 \).
    \item \textbf{Population from Density}: Population = Density \( \times \) Area.
    A town has a population density of 1000 people per square mile. If a circular region has an area of 4 square miles, how many people live in that region?
    Population = \( 1000 \frac{\text{people}}{\text{mile}^2} \times 4 \text{ miles}^2 = \underline{4000} \) people.
    \item \textbf{Solving for Radius}: A circular delivery area needs to serve 12,000 people, and the population density is 1500 people per square mile.
    \begin{enumerate}[label=\alph*)]
        \item Find the area needed: Area = Population / Density.
        Area = \( \frac{12000 \text{ people}}{1500 \text{ people/mile}^2} = \underline{8} \text{ miles}^2 \).
        \item Solve for the radius using \( A = \pi r^2 \):
        \( 8 = \pi r^2 \)
        \( r^2 = \frac{8}{\pi} \)
        \( r = \sqrt{\frac{8}{\pi}} \approx \sqrt{\frac{8}{3.14159}} \approx \sqrt{2.546} \approx \underline{1.60} \) miles.
    \end{enumerate}
    \item \textbf{Applying to the Original Problem}: Restaurant wants to deliver to 30,000 people. Population density is 1200 people per square mile.
    \begin{enumerate}[label=\alph*)]
        \item Calculate the necessary area:
        Area = \( \frac{\text{Target Population}}{\text{Population Density}} = \frac{30000 \text{ people}}{1200 \text{ people/mile}^2} = \underline{25} \text{ miles}^2 \).
        \item Find the radius of the delivery area using \( A = \pi r^2 \):
        \( 25 = \pi r^2 \)
        \( r^2 = \frac{25}{\pi} \)
        \( r = \sqrt{\frac{25}{\pi}} = \frac{5}{\sqrt{\pi}} \approx \frac{5}{1.77245} \approx \underline{2.8209} \) miles.
        \item Round to one decimal place: \( r \approx \underline{2.8} \) miles.
        Compare to the choices: 2.8 miles, 5.0 miles, 1.6 miles, 8.0 miles. (Matches A).
    \end{enumerate}
\end{enumerate}

% Section for Question 14
\subsection*{Scaffolded Question for Assessment Item 14: Simplifying Radicals and Exponents}
The original question asks to simplify \( \sqrt{8} + \sqrt{32} - 2^{\frac{3}{2}} \). The following questions build skills in simplifying radicals and exponential expressions.

\begin{enumerate}[label=14.\arabic*]
    \item \textbf{Simplifying a Single Radical}: Simplify \( \sqrt{18} \) by factoring the number under the square root into a perfect square times another factor.
    \( \sqrt{18} = \sqrt{9 \times 2} = \sqrt{9} \times \sqrt{2} = \underline{3\sqrt{2}} \).
    \item \textbf{Combining Like Radicals}: Simplify the expression \( \sqrt{12} + \sqrt{48} \).
    First, simplify each square root:
    \( \sqrt{12} = \sqrt{4 \times 3} = 2\sqrt{3} \)
    \( \sqrt{48} = \sqrt{16 \times 3} = 4\sqrt{3} \)
    Then, combine like terms: \( 2\sqrt{3} + 4\sqrt{3} = (2+4)\sqrt{3} = \underline{6\sqrt{3}} \).
    \item \textbf{Understanding Fractional Exponents}: Evaluate \( 3^{\frac{3}{2}} \).
    Rewrite using the property \( a^{\frac{m}{n}} = (\sqrt[n]{a})^m \) or \( \sqrt[n]{a^m} \).
    \( 3^{\frac{3}{2}} = (\sqrt{3})^3 = \sqrt{3} \times \sqrt{3} \times \sqrt{3} = 3\sqrt{3} \).
    Alternatively, \( 3^{\frac{3}{2}} = \sqrt{3^3} = \sqrt{27} = \sqrt{9 \times 3} = 3\sqrt{3} \).
    The value is \underline{\(3\sqrt{3}\)}.
    \item \textbf{Applying to the Original Expression}: Simplify \( \sqrt{8} + \sqrt{32} - 2^{\frac{3}{2}} \).
    \begin{enumerate}[label=\alph*)]
        \item Simplify \( \sqrt{8} \): \( \sqrt{4 \times 2} = \underline{2\sqrt{2}} \).
        \item Simplify \( \sqrt{32} \): \( \sqrt{16 \times 2} = \underline{4\sqrt{2}} \).
        \item Evaluate \( 2^{\frac{3}{2}} \): \( (\sqrt{2})^3 = \sqrt{2} \times \sqrt{2} \times \sqrt{2} = \underline{2\sqrt{2}} \).
        \item Combine the results: \( 2\sqrt{2} + 4\sqrt{2} - 2\sqrt{2} = (2+4-2)\sqrt{2} = \underline{4\sqrt{2}} \).
        Compare to the choices. (Matches C).
    \end{enumerate}
\end{enumerate}

% Section for Question 15
\subsection*{Scaffolded Question for Assessment Item 15: Inverse Variation}
The original question involves inverse variation where \( M \) varies inversely with \( x \), with \( M = 2 \) when \( x = 10 \), and asks for \( M \) when \( x = 5 \). The following questions build understanding of inverse variation.

\begin{enumerate}[label=15.\arabic*]
    \item \textbf{Understanding Inverse Variation}: If \( y \) varies inversely with \( x \), the relationship is \( y = \frac{k}{x} \), where \(k\) is the constant of variation.
    If \( y = 6 \) when \( x = 4 \), find \( k \).
    \( 6 = \frac{k}{4} \implies k = 6 \times 4 = \underline{24} \).
    \item \textbf{Finding a New Value}: Using the relationship \( y = \frac{k}{x} \), with \( k = 12 \) (from a different problem), calculate \( y \) when \( x = 3 \).
    \( y = \frac{12}{3} = \underline{4} \).
    \item \textbf{Setting Up the Equation and Solving}: If \( M \) varies inversely with \( x \), and \( M = 5 \) when \( x = 8 \):
    \begin{enumerate}[label=\alph*)]
        \item Write the inverse variation equation by finding \( k \):
        \( M = \frac{k}{x} \implies 5 = \frac{k}{8} \implies k = 5 \times 8 = \underline{40} \).
        So the equation is \( M = \frac{40}{x} \).
        \item Then, find \( M \) when \( x = 4 \):
        \( M = \frac{40}{4} = \underline{10} \).
    \end{enumerate}
    \item \textbf{Applying to the Original Problem}: Given \( M \) varies inversely with \( x \), and \( M = 2 \) when \( x = 10 \).
    \begin{enumerate}[label=\alph*)]
        \item Find the constant \( k \):
        \( M = \frac{k}{x} \implies 2 = \frac{k}{10} \implies k = 2 \times 10 = \underline{20} \).
        The equation is \( M = \frac{20}{x} \).
        \item Calculate \( M \) when \( x = 5 \):
        \( M = \frac{20}{5} = \underline{4} \).
    \end{enumerate}
\end{enumerate}

% Section for Question 32
\subsection*{Scaffolded Question for Assessment Item 32: Analyzing Expression Behavior}
The original question asks which statements result in \( 2x^2 + 3 + \frac{7}{y} \) increasing, for \( x, y > 0 \). The following questions build understanding of expression behavior.

\begin{enumerate}[label=32.\arabic*]
    \item \textbf{Term Analysis}: For \( 2x^2 + 3 + \frac{7}{y} \) (given \(x > 0, y > 0\)):
    \begin{enumerate}[label=\alph*)]
        \item Term \( 2x^2 \): As \( x \) increases, \(x^2\) increases, so \(2x^2\) \underline{increases}.
        \item Term \( \frac{7}{y} \): As \( y \) increases, the denominator increases, so the fraction \( \frac{7}{y} \) \underline{decreases}.
        \item Why does \( \frac{7}{y} \) decrease as \(y\) increases (for \(y>0\))? \underline{You are dividing a constant positive number by a larger positive number, resulting in a smaller positive value.}
    \end{enumerate}
    \item \textbf{Effect of Single Changes}: Consider the expression \( E = 2x^2 + 3 + \frac{7}{y} \).
    Start with \( x = 1 \), \( y = 2 \). Value = \( 2(1)^2 + 3 + \frac{7}{2} = 2 + 3 + 3.5 = 8.5 \).
    \begin{enumerate}[label=\alph*)]
        \item Change \( x \) to 2 (increases), keep \( y = 2 \) (constant):
        New value = \( 2(2)^2 + 3 + \frac{7}{2} = 2(4) + 3 + 3.5 = 8 + 3 + 3.5 = 14.5 \).
        The expression \underline{increases} (from 8.5 to 14.5).
        \item Change \( y \) to 4 (increases), keep \( x = 1 \) (constant):
        New value = \( 2(1)^2 + 3 + \frac{7}{4} = 2 + 3 + 1.75 = 6.75 \).
        The expression \underline{decreases} (from 8.5 to 6.75).
    \end{enumerate}
    \item \textbf{Effect of Combined Changes (Example)}: For a different expression \( E_2 = x^2 + 1 - \frac{5}{y} \) (\(x,y > 0\)), test scenarios:
    \begin{enumerate}[label=\alph*)]
        \item \( x \) increases, \( y \) decreases:
        \(x^2\) term increases. \(-\frac{5}{y}\) term: as \(y\) decreases, \(\frac{5}{y}\) increases, so \(-\frac{5}{y}\) decreases.
        Effect is ambiguous without knowing magnitudes.
        Let's re-evaluate for the expression \(E_3 = x^2 + 1 + \frac{5}{y}\)
        If \(x\) increases, \(x^2\) increases \(\implies\) \(E_3\) tends to increase.
        If \(y\) decreases, \(\frac{5}{y}\) increases \(\implies\) \(E_3\) tends to increase.
        So if \( x \) increases and \( y \) decreases, the expression \(E_3\) \underline{increases}.
        \item For \(E_3 = x^2 + 1 + \frac{5}{y}\): If \( x \) decreases and \( y \) increases:
        \(x^2\) decreases \(\implies\) \(E_3\) tends to decrease.
        \(\frac{5}{y}\) decreases \(\implies\) \(E_3\) tends to decrease.
        So if \( x \) decreases and \( y \) increases, the expression \(E_3\) \underline{decreases}.
    \end{enumerate}
    \item \textbf{Applying to the Original Problem}: For \( E = 2x^2 + 3 + \frac{7}{y} \) (\(x, y > 0\)). Which changes make \(E\) increase?
    (Term \(2x^2\) increases if \(x\) increases, decreases if \(x\) decreases. Term \(3\) is constant. Term \(\frac{7}{y}\) increases if \(y\) decreases, decreases if \(y\) increases.)
    \begin{enumerate}[label=\alph*)]
        \item A. \( x \) decreasing and \( y \) increasing:
        \(2x^2\) decreases. \(\frac{7}{y}\) decreases. So \(E\) \underline{decreases}.
        \item B. \( x \) increasing and \( y \) decreasing:
        \(2x^2\) increases. \(\frac{7}{y}\) increases. So \(E\) \underline{increases}. (Select B)
        \item C. \( y \) increasing and \( x \) remaining constant:
        \(2x^2\) constant. \(\frac{7}{y}\) decreases. So \(E\) \underline{decreases}.
        \item D. \( y \) decreasing and \( x \) remaining constant:
        \(2x^2\) constant. \(\frac{7}{y}\) increases. So \(E\) \underline{increases}. (Select D)
        \item E. \( x \) decreasing and \( y \) remaining constant:
        \(2x^2\) decreases. \(\frac{7}{y}\) constant. So \(E\) \underline{decreases}.
        \item F. \( x \) increasing and \( y \) remaining constant:
        \(2x^2\) increases. \(\frac{7}{y}\) constant. So \(E\) \underline{increases}. (Select F)
        \item Select all that apply: \underline{B, D, F}.
    \end{enumerate}
\end{enumerate}

\section*{Original Assessment Questions}

\subsection*{Question 3}
It takes Faucet A 8 hours to fill a tank, and it takes Faucet B 4 hours. If the tank is empty, how long will it take the two faucets to fill the tank together?
\[ \framebox[1.5cm]{\phantom{X}} \text{ hours and } \framebox[1.5cm]{\phantom{X}} \text{ minutes} \]

\subsection*{Question 10}
The formula \( N = S(P - V) - F \) represents net income \(N\), where \(P\) represents sales price, \(V\) is the variable cost per unit, \(S\) is the sales volume, and \(F\) are fixed costs. Complete the formula to find the variable cost per unit.
\[ \text{Formula for variable cost: } V = \framebox[1cm]{\phantom{X}} - \frac{\framebox[0.7cm]{\phantom{X}} + \framebox[0.7cm]{\phantom{X}}}{\framebox[0.7cm]{\phantom{X}}} \]
(Based on scaffold: \(V = P - \frac{N+F}{S}\). So blanks are P, N, F, S)

\subsection*{Question 13}
A pizza restaurant is located in a town with a population density of 1200 people per square mile. What delivery radius will allow the pizza restaurant to deliver to approximately 30,000 people?
\begin{enumerate}[label=\Alph*.]
    \item 2.8 miles
    \item 5.0 miles
    \item 1.6 miles
    \item 8.0 miles
\end{enumerate}

\subsection*{Question 14}
Simplify.
\[ \sqrt{8} + \sqrt{32} - 2^{\frac{3}{2}} \]
\begin{enumerate}[label=\Alph*.]
    \item \( -2\sqrt{2} - \sqrt[3]{2} \) % OCR issue, likely meant to be different.
    \item \( 8\sqrt{2} \)
    \item \( 4\sqrt{2} \)
    \item 0
\end{enumerate}
(Note: The OCR for A is unusual. The correct answer from scaffold is \(4\sqrt{2}\), which is C)

\subsection*{Question 15}
\(M\) varies inversely with \(x\). If \( M = 2 \) when \( x = 10 \), find the value of \(M\) when \( x = 5 \).
\[ M = \framebox[1.5cm]{\phantom{X}} \]

\subsection*{Question 32}
In the expression \( 2x^2 + 3 + \frac{7}{y} \), \(x\) and \(y\) are positive numbers. Select all the statements which result in the value of the expression increasing.
\begin{enumerate}[label=\Alph*.]
    \item[\XBox] \(x\) decreasing and \(y\) increasing
    \item[\XBox] \(x\) increasing and \(y\) decreasing
    \item[\XBox] \(y\) increasing and \(x\) remaining constant
    \item[\XBox] \(y\) decreasing and \(x\) remaining constant
    \item[\XBox] \(x\) decreasing and \(y\) remaining constant
    \item[\XBox] \(x\) increasing and \(y\) remaining constant
\end{enumerate}
(Note: Replace \XBox with \Square if you want empty boxes for students to fill)

% Ending the document
\end{document}