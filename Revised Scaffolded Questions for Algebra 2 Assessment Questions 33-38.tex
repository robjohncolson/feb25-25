\documentclass[12pt]{article}

% Setting up page geometry
\usepackage[margin=1in]{geometry}

% Including packages for mathematical typesetting
\usepackage{amsmath}
\usepackage{amssymb}
\usepackage{mathtools}

% Including package for enhanced enumeration
\usepackage{enumitem}

% Including package for better spacing and formatting
\usepackage{parskip}

% Setting up font: Latin Modern
\usepackage{lmodern}

% Document begins
\begin{document}

% Creating title
\begin{center}
    \textbf{Revised Scaffolded Questions for Algebra 2 Assessment (Questions 33--38)}
\end{center}

% Introduction
This document provides revised scaffolded questions to help students prepare for questions 33 through 38 of the enVision Algebra 2 Progress Monitoring Assessment Form C. Each question includes four scaffolded steps to build understanding from basic concepts to the level required by the assessment, with clear guidance for concept-naive students.

% Section for Question 33
\section*{Question 33: Linear vs Exponential Growth Models}
The original question involves modeling Lucia’s linear (12 residents/day) and Caleb’s exponential (4 people, each contacting 4 more daily) growth, combining them, applying a 35% voting rate, and calculating votes after 7 days. The following questions build understanding of linear and exponential models.

\begin{enumerate}[label=33.\arabic*]
    \item \textbf{Linear Models}: Linear functions \( f(x) = mx + b \) have a constant rate \( m \):
    \begin{enumerate}
        \item[a)] 10 contacts/day: \( f(x) = 10x \)
        \item[b)] 5 units/week, initial 20: \( f(x) = \_\_\_\_x + \_\_\_\_ \)
        \item[c)] Why constant rate? \_\_\_\_\_\_\_\_\_\_\_\_
    \end{enumerate}
    \item \textbf{Exponential Models}: Exponential functions \( f(x) = ab^x \) model multiplicative growth:
    \begin{enumerate}
        \item[a)] Triples daily, starts at 5: \( f(x) = 5 \cdot 3^x \)
        \item[b)] Starts at 2, each contacts 3 more daily: Day 1: 2, Day 2: 6, Day 3: 18 → \( f(x) = \_\_\_\_ \)
        \item[c)] Why cumulative growth? \_\_\_\_\_\_\_\_\_\_\_\_
    \end{enumerate}
    \item \textbf{Combining Models}: Add linear and exponential contributions:
    \begin{enumerate}
        \item[a)] Linear: 8x, Exponential: \( 2^x \): \( g(x) = 8x + 2^x \)
        \item[b)] Linear: 10x, Exponential: \( 3^x \), 30% effective: \( h(x) = 0.3(10x + 3^x) \)
        \item[c)] For \( x = 2 \): \( h(2) = \_\_\_\_ \)
    \end{enumerate}
    \item \textbf{Applying to the Original Problem}: Lucia: 12x, Caleb: \( 4^x \), 35% vote:
    \begin{enumerate}
        \item[a)] Combined: \( g(x) = 12x + 4^x \)
        \item[b)] Votes: \( h(x) = 0.35(12x + 4^x) \)
        \item[c)] For \( x = 7 \): \( 12 \cdot 7 = 84 \), \( 4^7 = 16384 \), \( h(7) = 0.35(84 + 16384) \approx \_\_\_\_ \)
    \end{enumerate}
\end{enumerate}

% Section for Question 34
\section*{Question 34: Rational Equations and Extraneous Solutions}
The original question asks to solve \( \frac{x^2 + 4}{x - 1} = \frac{5}{x - 1} \) and identify extraneous solutions. The following questions build understanding of rational equations.

\begin{enumerate}[label=34.\arabic*]
    \item \textbf{Basic Rational Equations}: If denominators are equal, equate numerators:
    \begin{enumerate}
        \item[a)] \( \frac{x + 2}{x - 3} = \frac{5}{x - 3} \): \( x + 2 = 5 \), \( x = 3 \). \\
        Check: \( x = 3 \) makes denominator zero (extraneous).
        \item[b)] \( \frac{2x}{x + 1} = \frac{4}{x + 1} \): \( x = \_\_\_\_ \). Check: \_\_\_\_.
    \end{enumerate}
    \item \textbf{Extraneous Solutions}: Solutions making denominators zero are extraneous:
    \begin{enumerate}
        \item[a)] \( \frac{x}{x - 4} = \frac{2}{x - 4} \): Extraneous: \( x = \_\_\_\_ \)
        \item[b)] Solve: \( x = 2 \). Check: \_\_\_\_.
    \end{enumerate}
    \item \textbf{Solving Equations}: Solve \( \frac{x^2 + 1}{x - 2} = \frac{3}{x - 2} \):
    \begin{enumerate}
        \item[a)] Restriction: \( x \neq 2 \)
        \item[b)] Equate: \( x^2 + 1 = 3 \), \( x^2 = 2 \), \( x = \pm \sqrt{2} \)
        \item[c)] Check: Neither \( \sqrt{2} \) nor \( -\sqrt{2} \) equals 2 (valid).
    \end{enumerate}
    \item \textbf{Applying to the Original Problem}: Solve \( \frac{x^2 + 4}{x - 1} = \frac{5}{x - 1} \):
    \begin{enumerate}
        \item[a)] Restriction: \( x \neq 1 \)
        \item[b)] Equate: \( x^2 + 4 = 5 \), \( x^2 = 1 \), \( x = \pm 1 \)
        \item[c)] Check: \( x = 1 \) is extraneous, \( x = -1 \) is valid.
        \item[d)] Answer: \( x = -1 \), extraneous: \( x = 1 \).
    \end{enumerate}
\end{enumerate}

% Section for Question 35
\section*{Question 35: Discontinuities in Rational Functions}
The original question asks where discontinuities occur in \( f(x) = \frac{x^2 + 5x}{x^2 - 2x - 35} \). The following questions build understanding of discontinuities.

\begin{enumerate}[label=35.\arabic*]
    \item \textbf{Factoring Quadratics}: Factor to find zeros:
    \begin{enumerate}
        \item[a)] \( x^2 - 6x + 8 = (x - 2)(x - 4) \): Zeros: \( x = 2, 4 \)
        \item[b)] \( x^2 + 3x - 10 \): Zeros: \_\_\_\_, \_\_\_\_
    \end{enumerate}
    \item \textbf{Finding Discontinuities}: Discontinuities occur where denominator = 0:
    \begin{enumerate}
        \item[a)] \( f(x) = \frac{x}{x - 5} \): Discontinuity: \( x = 5 \)
        \item[b)] \( f(x) = \frac{1}{x^2 - 4} \): Discontinuities: \_\_\_\_, \_\_\_\_
        \item[c)] Why discontinuities? \_\_\_\_\_\_\_\_\_\_\_\_
    \end{enumerate}
    \item \textbf{Removable Discontinuities}: Common factors may cancel:
    \begin{enumerate}
        \item[a)] \( f(x) = \frac{x - 1}{x^2 - x} = \frac{x - 1}{x(x - 1)} \): Cancel \( x - 1 \), discontinuity at \( x = 0 \)
        \item[b)] \( f(x) = \frac{x + 2}{x^2 + 5x + 6} \): Discontinuities: \_\_\_\_, \_\_\_\_.
    \end{enumerate}
    \item \textbf{Applying to the Original Problem}: For \( f(x) = \frac{x^2 + 5x}{x^2 - 2x - 35} \):
    \begin{enumerate}
        \item[a)] Numerator: \( x(x + 5) \), zeros: \( x = 0, -5 \)
        \item[b)] Denominator: \( (x - 7)(x + 5) \), zeros: \( x = 7, -5 \)
        \item[c)] Discontinuities: \( x = 7, -5 \). At \( x = -5 \), cancel \( x + 5 \), but original function undefined. Answer: \( x = -5, 7 \).
    \end{enumerate}
\end{enumerate}

% Section for Question 36
\section*{Question 36: Set Operations and Probability}
The original question asks whether the winning outcomes (odd number or 6) are the union, intersection, or complement of \( \{1, 2, 3, 5, 6\} \) and \( \{1, 3, 4, 5, 6\} \). The following questions build understanding of set operations.

\begin{enumerate}[label=36.\arabic*]
    \item \textbf{Set Operations}: For \( A = \{1, 3, 5\} \), \( B = \{2, 4, 6\} \):
    \begin{enumerate}
        \item[a)] Union: \( A \cup B = \{1, 2, 3, 4, 5, 6\} \)
        \item[b)] Intersection: \( A \cap B = \emptyset \)
        \item[c)] Why use union in probability? \_\_\_\_\_\_\_\_\_\_\_\_
    \end{enumerate}
    \item \textbf{Probability Context}: Dice roll, win on 1 or 2:
    \begin{enumerate}
        \item[a)] Set: \( \{1, 2\} \)
        \item[b)] Outcomes: \( \{1, 2, 3, 4, 5, 6\} \). Win = \( \{1, 2\} \).
    \end{enumerate}
    \item \textbf{Analyzing Sets}: For \( \{1, 3, 5\} \), \( \{1, 2, 5\} \):
    \begin{enumerate}
        \item[a)] Union: \( \{1, 2, 3, 5\} \)
        \item[b)] Intersection: \( \{1, 5\} \)
        \item[c)] Practice: If win on 1 or 5, which operation? \_\_\_\_.
    \end{enumerate}
    \item \textbf{Applying to the Original Problem}: Win on odd or 6:
    \begin{enumerate}
        \item[a)] Winning: \( \{1, 3, 5, 6\} \)
        \item[b)] Sets: \( \{1, 2, 3, 5, 6\} \), \( \{1, 3, 4, 5, 6\} \)
        \item[c)] Intersection: \( \{1, 3, 5, 6\} \). Answer: Intersection.
    \end{enumerate}
\end{enumerate}

% Section for Questions 37-38
\section*{Questions 37--38: Conditional Probability and Data Analysis}
The original questions involve calculating P(heavy metal | 12th grade) and comparing P(10th grade | rock) vs. P(rock | 10th grade) using a two-way table. The following questions build understanding of conditional probability.

\begin{enumerate}[label=37.\arabic*]
    \item \textbf{Reading Tables}: 
    \[
    \begin{array}{c|ccc|c}
    & \text{Rock} & \text{Hip-Hop} & \text{Heavy Metal} & \text{Total} \\
    \hline
    10\text{th} & 16 & 12 & 4 & 32 \\
    11\text{th} & 18 & 10 & 12 & 40 \\
    12\text{th} & 16 & 8 & 6 & 30 \\
    \hline
    \text{Total} & 50 & 30 & 22 & 102 \\
    \end{array}
    \]
    \begin{enumerate}
        \item[a)] 11th graders, hip-hop: 10
        \item[b)] Total 10th graders: \_\_\_\_
        \item[c)] Why use totals? \_\_\_\_\_\_\_\_\_\_\_\_
    \end{enumerate}
    \item \textbf{Basic Probability}: \( P(\text{event}) = \frac{\text{favorable}}{\text{total}} \):
    \begin{enumerate}
        \item[a)] P(11th grade): \( \frac{40}{102} \)
        \item[b)] P(rock | 11th grade): \( \frac{18}{40} = \frac{9}{20} \)
    \end{enumerate}
    \item \textbf{Conditional Probability}: \( P(A|B) = \frac{\text{A and B}}{\text{B}} \):
    \begin{enumerate}
        \item[a)] P(hip-hop | 10th grade): \( \frac{12}{32} = \frac{3}{8} \)
        \item[b)] P(10th grade | hip-hop): \( \frac{12}{30} = \frac{2}{5} \). Compare: \_\_\_\_.
    \end{enumerate}
    \item \textbf{Applying to the Original Problems}:
    \begin{enumerate}
        \item[a)] Q37: P(heavy metal | 12th grade): \( \frac{6}{30} = \frac{1}{5} = 20\% \)
        \item[b)] Q38: P(10th grade | rock): \( \frac{16}{50} = 0.32 \). \\
        P(rock | 10th grade): \( \frac{16}{32} = 0.5 \). \\
        0.32 < 0.5, so P(10th grade | rock) is \_\_\_\_ P(rock | 10th grade).
    \end{enumerate}
\end{enumerate}

% Ending the document
\end{document}