\documentclass[12pt]{article}

% Setting up page geometry
\usepackage[margin=1in]{geometry}

% Including packages for mathematical typesetting
\usepackage{amsmath}
\usepackage{amssymb}
\usepackage{mathtools}

% Including package for enhanced enumeration
\usepackage{enumitem}

% Including package for better spacing and formatting
\usepackage{parskip}

% Setting up font: Latin Modern
\usepackage{lmodern}

% Document begins
\begin{document}

% Creating title
\begin{center}
    \textbf{Revised Scaffolded Questions for Algebra 2 Assessment (Questions 17--20)}
\end{center}

% Introduction
This document provides revised scaffolded questions to help students prepare for questions 17 through 20 of the enVision Algebra 2 Progress Monitoring Assessment Form C. Each question includes four scaffolded steps to build understanding from basic concepts to the level required by the assessment, with clear guidance for concept-naive students.

% Section for Question 17
\section*{Question 17: Factoring Quadratics and Finding Zeros}
The original question asks to factor \( x^2 - 33x + 32 \) to find the zeros of \( f(x) = x^2 - 33x + 32 \). The following questions build understanding of factoring quadratics.

\begin{enumerate}[label=17.\arabic*]
    \item \textbf{Basic Factoring}: Factor by finding two numbers that multiply to the constant term and add to the middle coefficient:
    \begin{enumerate}
        \item[a)] \( x^2 + 7x + 10 \): Numbers multiply to 10, add to 7: 2, 5. \\
        Factored: \( (x + 2)(x + 5) \)
        \item[b)] \( x^2 - 9x + 20 \): Numbers multiply to 20, add to -9: -4, -5. \\
        Factored: \( (x - 4)(x - 5) \)
        \item[c)] Why does factoring find zeros? \_\_\_\_\_\_\_\_\_\_\_\_
    \end{enumerate}
    \item \textbf{Finding Zeros}: Find zeros by setting factors to zero:
    \begin{enumerate}
        \item[a)] \( f(x) = (x - 3)(x + 6) \): Zeros: \( x = 3 \), \( x = -6 \)
        \item[b)] \( f(x) = (x - 2)(x - 8) \): Zeros: \( x = \_\_\_\_ \), \( x = \_\_\_\_ \)
        \item[c)] Verify one zero: For \( x = 2 \), compute \( f(2) = (2 - 2)(2 - 8) = \_\_\_\_ \).
    \end{enumerate}
    \item \textbf{Larger Coefficients}: Factor \( x^2 - 14x + 45 \):
    \begin{enumerate}
        \item[a)] Factors of 45: \( 1 \times 45 \), \( 3 \times 15 \), \( 5 \times 9 \). \\
        Add to -14: -5, -9. \\
        Factored: \( (x - 5)(x - 9) \)
        \item[b)] Zeros: \( x = 5 \), \( x = 9 \)
        \item[c)] Practice: Factor \( x^2 - 16x + 60 \): Numbers: \_\_\_\_, \_\_\_\_. \\
        Factored: \_\_\_\_. Zeros: \_\_\_\_, \_\_\_\_.
    \end{enumerate}
    \item \textbf{Applying to the Original Problem}: Factor \( x^2 - 33x + 32 \):
    \begin{enumerate}
        \item[a)] Factors of 32: \( 1 \times 32 \), \( 2 \times 16 \), \( 4 \times 8 \). \\
        Add to -33: \_\_\_\_, \_\_\_\_.
        \item[b)] Factored: \( (x \_\_\_\_)(x \_\_\_\_) \)
        \item[c)] Zeros: \( x = \_\_\_\_ \), \( x = \_\_\_\_ \)
    \end{enumerate}
\end{enumerate}

% Section for Question 18
\section*{Question 18: Arithmetic Sequences}
The original question involves determining if the sequence (Monday: 240, Tuesday: 290, Friday: 440) is arithmetic and predicting Saturday’s attendance. The following questions build understanding of arithmetic sequences.

\begin{enumerate}[label=18.\arabic*]
    \item \textbf{Identifying Arithmetic Sequences}: A sequence is arithmetic if differences between consecutive terms are constant:
    \begin{enumerate}
        \item[a)] \( 4, 7, 10, 13, \ldots \): Differences: \( 7 - 4 = 3 \), \( 10 - 7 = 3 \). \\
        Arithmetic? Yes. Common difference: \( d = 3 \).
        \item[b)] \( 8, 6, 4, 2, \ldots \): Differences: \( 6 - 8 = \_\_\_\_ \), \( 4 - 6 = \_\_\_\_ \). \\
        Arithmetic? \_\_\_\_. Common difference: \_\_\_\_.
        \item[c)] Why constant differences? \_\_\_\_\_\_\_\_\_\_\_\_
    \end{enumerate}
    \item \textbf{Finding Common Differences}: Given festival attendance:
    \begin{enumerate}
        \item[a)] Monday = 200, Tuesday = 250: \( d = 250 - 200 = \_\_\_\_ \)
        \item[b)] Monday = 240, Tuesday = 290: \( d = \_\_\_\_ \)
        \item[c)] If Wednesday = 340, check: \( 340 - 290 = \_\_\_\_. Is it consistent? \_\_\_\_.
    \end{enumerate}
    \item \textbf{Recursive Formulas}: For an arithmetic sequence, \( a_n = a_{n-1} + d \):
    \begin{enumerate}
        \item[a)] Sequence: \( 5, 9, 13, 17, \ldots \): \( a_1 = 5 \), \( d = 4 \). \\
        Formula: \( a_1 = 5 \), \( a_n = a_{n-1} + 4 \)
        \item[b)] Sequence: 240, 290, 340, \ldots: \( a_1 = \_\_\_\_ \), \( d = \_\_\_\_. \\
        Formula: \_\_\_\_.
    \end{enumerate}
    \item \textbf{Applying to the Original Problem}: Monday = 240, Tuesday = 290, Friday = 440:
    \begin{enumerate}
        \item[a)] Common difference: \( d = 290 - 240 = \_\_\_\_ \).
        \item[b)] Recursive formula: \( a_1 = \_\_\_\_ \), \( a_n = \_\_\_\_ \).
        \item[c)] Predict terms: Wednesday = \_\_\_\_, Thursday = \_\_\_\_, Friday = \_\_\_\_. \\
        Check Friday: Matches 440? \_\_\_\_.
        \item[d)] Saturday: \_\_\_\_ people.
    \end{enumerate}
\end{enumerate}

% Section for Question 19
\section*{Question 19: Solving Equations Graphically}
The original question asks to solve \( (x - 2)^2 - 1 = (x - 2)^3 + 1 \) graphically. The following questions build understanding of graphical solutions.

\begin{enumerate}[label=19.\arabic*]
    \item \textbf{Simple Graphical Solutions}: Solve \( x + 2 = 5 \):
    \begin{enumerate}
        \item[a)] Graph: \( y = x + 2 \), \( y = 5 \).
        \item[b)] Intersection: \( (3, 5) \). Solution: \( x = 3 \).
        \item[c)] Practice: Solve \( 3x = 9 \): Intersection: \_\_\_\_. Solution: \_\_\_\_.
    \end{enumerate}
    \item \textbf{Quadratic Equations}: Solve \( (x - 1)^2 = 9 \):
    \begin{enumerate}
        \item[a)] Graph: \( y = (x - 1)^2 \), \( y = 9 \).
        \item[b)] Intersections: \( (-2, 9) \), \( (4, 9) \). Solutions: \( x = -2 \), \( x = 4 \).
        \item[c)] Practice: Solve \( x^2 = 4 \): Solutions: \_\_\_\_, \_\_\_\_.
    \end{enumerate}
    \item \textbf{Complex Functions}: Solve \( (x - 1)^2 = x - 1 \):
    \begin{enumerate}
        \item[a)] Graph: \( y = (x - 1)^2 \), \( y = x - 1 \).
        \item[b)] Move to one side: \( (x - 1)^2 - (x - 1) = 0 \). \\
        Factor: \( (x - 1)(x - 2) = 0 \). Zeros: \( x = 1 \), \( x = 2 \).
        \item[c)] Practice: Solve \( (x - 1)^2 = 2x - 2 \): Zeros: \_\_\_\_, \_\_\_\_.
    \end{enumerate}
    \item \textbf{Applying to the Original Problem}: Solve \( (x - 2)^2 - 1 = (x - 2)^3 + 1 \):
    \begin{enumerate}
        \item[a)] Set: \( (x - 2)^2 - 1 - (x - 2)^3 - 1 = 0 \).
        \item[b)] Simplify: \( (x - 2)^2 - (x - 2)^3 - 2 = 0 \).
        \item[c)] Let \( u = x - 2 \): \( u^2 - u^3 - 2 = 0 \). \\
        Graph \( y = u^2 - u^3 - 2 \). Find zeros: Test \( u = 1 \): \( 1 - 1 - 2 = -2 \). \\
        Try numerically or graphically to find \( u \approx -1.52 \).
        \item[d)] Solve: \( x - 2 \approx -1.52 \), so \( x \approx 0.48 \).
    \end{enumerate}
\end{enumerate}

% Section for Question 20
\section*{Question 20: Completing the Square}
The original question asks for the constant to add to both sides of \( 3x^2 + 4x = 5 \) to complete the square. The following questions build understanding of completing the square.

\begin{enumerate}[label=20.\arabic*]
    \item \textbf{Perfect Square Trinomials}: Complete to form a perfect square:
    \begin{enumerate}
        \item[a)] \( x^2 + 10x + \_\_\_\_ = (x + 5)^2 \): Half of 10: 5, squared: 25.
        \item[b)] \( x^2 - 6x + \_\_\_\_ = (x - \_\_\_\_)^2 \): Half of -6: -3, squared: 9.
        \item[c)] \( x^2 + 12x + \_\_\_\_ = (x + \_\_\_\_)^2 \): \_\_\_\_, \_\_\_\_.
    \end{enumerate}
    \item \textbf{Completing the Square (\( a = 1 \))}: For \( x^2 + 8x = 3 \):
    \begin{enumerate}
        \item[a)] Half of 8: 4, squared: 16.
        \item[b)] Add: \( x^2 + 8x + 16 = 3 + 16 \).
        \item[c)] Factor: \( (x + 4)^2 = 19 \).
        \item[d)] Practice: For \( x^2 + 10x = 6 \): Constant: \_\_\_\_. Result: \_\_\_\_.
    \end{enumerate}
    \item \textbf{Completing the Square (\( a \neq 1 \))}: For \( 2x^2 + 12x = 8 \):
    \begin{enumerate}
        \item[a)] Factor: \( 2(x^2 + 6x) = 8 \).
        \item[b)] Complete inside: Half of 6: 3, squared: 9. \\
        \( 2(x^2 + 6x + 9) = 8 + 2 \cdot 9 = 26 \).
        \item[c)] Simplify: \( 2(x + 3)^2 = 26 \).
        \item[d)] Constant added to right: \( 2 \cdot 9 = 18 \).
        \item[e)] Practice: For \( 4x^2 + 8x = 12 \): Constant: \_\_\_\_. Result: \_\_\_\_.
    \end{enumerate}
    \item \textbf{Applying to the Original Problem}: For \( 3x^2 + 4x = 5 \):
    \begin{enumerate}
        \item[a)] Factor: \( 3(x^2 + \frac{4}{3}x) = 5 \).
        \item[b)] Complete: Half of \( \frac{4}{3} \): \( \frac{2}{3} \), squared: \( \frac{4}{9} \). \\
        \( 3\left(x^2 + \frac{4}{3}x + \frac{4}{9}\right) = 5 + 3 \cdot \frac{4}{9} \).
        \item[c)] Simplify: \( 3\left(x + \frac{2}{3}\right)^2 = 5 + \frac{12}{9} = 5 + \frac{4}{3} = \frac{19}{3} \).
        \item[d)] Constant added: \( 3 \cdot \frac{4}{9} = \frac{4}{3} \). Matches choice (B).
    \end{enumerate}
\end{enumerate}

% Ending the document
\end{document}